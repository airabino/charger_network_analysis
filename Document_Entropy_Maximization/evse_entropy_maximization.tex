\section{\gls{evse} Entropy Maximization}

\gls{evse} location optimization is a topic which has produced, continues to produce, and will continue to produce ludicrous volumes of research very little of which informs deployment in any real way. This situation is the result of the fact that \gls{evse} infrastructure is insufficient in many ways and is perceived to serve as a impediment to \gls{pev} sales growth meaning that the large body of optimization and data analysis focused researchers are able to fund projects and publish papers related to the topic. In the most general sense, optimization of \gls{evse} locations is not an especially unique problem and most seek to optimize the allocation of scarce development resources to minimize a given cost function. A common theme of these papers is the goal of optimizing efficiency in metrics which are either business or consumer oriented. Robustness is often treated as a constraint if considered at all. This line of thinking has led to large concentrations of \gls{evse} in those areas most dense with \gls{pev} ownership and a number of subsidized charging stations located at arbitrary intervals along major transportation corridors. Evenly spaced chargers along corridors is not necessarily a bad solution but it is an arbitrary one.

Consider, instead, how redundancy might be computed from and informational perspective. As a driver embarks on a long trip (one that requires charging), the driver will have a charging window. The charging window starts at a distance where a substantial portion of the vehicle's range is used and will end when the driver is close enough to the destination that additional range is not required. For example, if a car has 300 km of range and is undertaking a 400 km trip at least one charge is required. However there will be little region to charge in the first 100 km because doing so will mean having to charge twice. For the same reason the last 100 km is unlikely to see a charge event. Thus, the charging window for the trip is the 200 km in the middle. If chargers were evenly spaced in 50 km intervals along the driver's route then of the 7 chargers available the middle 5 would be viable options. Since the driver only needs to charge once there is real redundancy built in. At the beginning of the trip there are 5 shortest paths that the driver must choose between. If the driver passes the first middle charger without charging then the number of shortest paths goes from 5 to 4, if the driver charges at the first middle charger then the number of shortest paths goes from 5 to 1. In this scenario charging provides more information than not charging and this is not surprising, stopping to charge will be less common than passing a charger.

In the above example the trip origin and destination as well as the vehicle range are known allowing for the definition of a charging window. In practice, these quantities can only be imputed stochastically. The basic principle, however, can be generalized:

\begin{lemma}[Charger-Charger Edge Energy Cost Upper Limit]
	Under normal circumstances, vehicles will not travel from charger $u$ to charger $v$ if the edge energy cost $d(u, v)$ is greater than a given upper limit $L^U$ which is less than or equal to the vehicle's maximum energy storage capability.
\end{lemma}

\begin{corollary}
	Chargers $u$ and $v$ are incident if and only if $d(u, v) < L^U$.
\end{corollary}

\begin{lemma}[Charger-Charger Edge Energy Cost Lower Limit]
	Under normal circumstances, vehicles will not travel from charger $u$ to charger $v$ if the edge energy cost $d(u, v)$ is less than a given lower limit $L^L$ which is greater than or equal to the vehicle's minimum energy storage capability.
\end{lemma}

\begin{corollary}
	Chargers $u$ and $v$ are incident if and only if $d(u, v) > L^L$.
\end{corollary}

\begin{lemma}[Charger-Place Edge Energy Cost Upper Limit]
	Under normal circumstances, vehicles will not travel from charger $u$ to place $v$ if the edge energy cost $d(u, v)$ is greater than a given upper limit $L^L$ which is less than or equal to the vehicle's minimum energy storage capability. The same holds for place-charger edges.
\end{lemma}

\begin{corollary}
	Charger $u$ and place $v$ are incident if and only if $d(u, v) < L^U$.
\end{corollary}

\begin{lemma}[Place-Place Edge Energy Cost Upper Limit]
	Under normal circumstances, vehicles will not travel from place $u$ to place $v$ if the edge energy cost $d(u, v)$ is greater than a given upper limit $L^L$ which is less than or equal to the vehicle's minimum energy storage capability.
\end{lemma}

\begin{corollary}
	Place $u$ and place $v$ are incident if and only if $d(u, v) < L^U$.
\end{corollary}

In other words, chargers relationships with each other are determined by distance. If two chargers are very far apart then it is unlikely that a car will travel from one to the other without having to charge at a third location. If two chargers are close together a vehicle may use either on a given trip but is unlikely to use both. The constants $L^L$ and $L^U$ define the span of incidence. Chargers themselves serve to allow for trips between places and, thus, will only be visited for the purposes of charging. Places can be visited at any \gls{soc} so there is no minimum edge cost for incidence if the edge connects at least one place node.
	