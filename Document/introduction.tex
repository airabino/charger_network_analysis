\section{Introduction}

When discussing transportation networks, one must consider several linked and mutually supporting networks including road, rail, marine, and air infrastructure in addition to those energy and information networks which allow for vehicles to function. Multiple support systems are required for the efficient and economical operation of vehicles which underpins modern economies and standards of living. The various networks are coincident at a subset of nodes but, otherwise, may be operated in various manners by dissimilar stakeholders in service of divergent objectives. none of the related networks is a subset of any other but all share dependence relationships which may be asymmetrical.

The mode of vehicles in today's transportation sector, whether passenger or cargo oriented, are \glspl{icev} but a growing number of these vehicles are being supplanted by \glspl{pev}. This shift necessitates a re-alignment of the networks underpinning travel as the nodes of interaction between physical infrastructure and the energy system are incompatible between \glspl{icev} and \glspl{bev}. \glspl{icev} receive energy from a fuel distribution network which connects extraction operations to processing plants and ultimately to distribution stations. The fueling network is inherently inefficient due to constraints on the location of extraction sites, the economies of scale and access related to processing sites, and the requirement to expend fuel to transport fuel to end user locations which are widely distributed. It is, nevertheless, the case that this network has been subject to tremendous amounts of capital expenditure and operational optimization such that it's function and economic viability is ensured as long as extricable fuel supplies continue to be discovered and fuel-burning vehicles continue to be used.

\glspl{pev}, by contrast, draw energy from the electricity distribution network which also powers static loads such as those from buildings. Also the recipient of substantial and continuous investment, the power grid has been, historically, optimized to meet demand from buildings via the utilization of dispatchable resources including fuel-burning power plants. As the number of \glspl{pev} continues to rise, the load required to power these vehicles will become a greater portion of the overall load required. The interaction between vehicles and the power grid is heavily bound by constraints at coincident nodes and within their valency. Limitations exist on power transfer rates between the charger and battery as well as between the charger and power generation as a function of the distribution grid.

The most salient difference between the \gls{icev} fuel delivery network and the \gls{evse} network is at the interaction points. There are many more fueling stations in any given region of a developed country than public chargers. On the other hand, there are many more private charging locations than private fueling locations. \glspl{pev} are able to charge at low rates from many nodes including homes and buildings which are connected to the grid. Low rate charging is preferable for several reasons including lower costs per kWh and lower battery degradation. High rate charging is generally reserved for situations wherein charge time is relevant such as on itineraries whose distance is in excess of a vehicle's single charge range or for daily use by people unable to charge at home/work. The multifaceted nature of \gls{pev} charging contrasts to that of \gls{icev} fueling which takes lace at public access stations and at similar rates for the vast majority of fueling events. A significant portion of \gls{pev} users can accomplish most charging without relying on public-access stations and these are encourages to maintain this behavior. The multi-tiered "charging pyramid" dictates that fewer public-access fast charging stations can be supported for a given number of \glspl{pev} compared to fueling stations and \glspl{icev}.

As a result of the economic realities of charging stations, increases in network cardinality have, generally, been driven by incentives. The goal of these incentives is to create a network of public-access charging stations which is sufficiently ubiquitous and reliable that potential buyers will view \glspl{pev} as completely viable vehicles. A secondary goal would be to incentivise an efficient distribution of these stations such that profitable operations can be maintained after the subsidy period runs out. This document outlines methods of analysis for these networks which are fundamental and extensible and provide insights which can be used for further roll-out.