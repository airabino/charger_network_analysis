\subsection*{Transportation Accessibility}

Transportation accessibility is a framework which encompasses demand factors such as land use and temporal availability, impedance factors such as transportation system design, and universal factors such as personal preference \cite{Geurs_2004}. Literature provides four essential frameworks for computing access as surveyed in \cite{Handy_1997, Kwan_1998, Geurs_2004, Miller_2018, Handy_2020}. Much of the variance between analyses is on the demand side. There will usually be several near-equivalents for any given opportunity type and computing a single-number metric of access requires a model.

The simplest methods for selecting opportunities are Proximity methods \cite{Wachs_1973, Vickerman_1974}. Proximity methods determine access by proximity to the closest relevant opportunity. Proximity methods do not account for heterogeneity within an opportunity category nor for the benefits of redundancy within an opportunity category. The inverse are Isocost methods which determine access by the number of opportunities available within a given isocost polygon. Isocost methods do not consider the differences in travel costs within the isocost region. Isocost methods have been used widely due to their transparency and computational lightness \cite{Easa_1993}, and form the basis for modern big-data methods such as the US DOE's Mobility Energy Productivity metric \cite{Hou_2019}.

Proximity and Isocost methods are easy to compute because they handle redundancy with arbitrary rules. In practice, equivalent and near-equivalent opportunities compete with one-another if sufficiently proximate or if the paths required to reach them overlap \cite{Stouffer_1940}. Gravity and Entropy methods \cite{Noulas_2012, Jung_2008} address this shortcoming by computing the cumulative effect of multiple opportunities on access for a given origin. Gravity/Entropy methods define accessibility as the intensity of the possibility for interaction \cite{Hansen_1959}. Implicit in the formulation of Gravity/Entropy methods is that every opportunity has some effect on every individual, even if negligible, and the effect of any one opportunity is determined by its network position. Discrete Choice Modeling \cite{Ben_Akiva_1985} is often used to explain revealed choices from traffic measurements and travel surveys \cite{Cevero_1995, Shen_1998, Karst_2003} in order to fit Gravity/Entropy models allowing for generalization and the evaluation of hypothetical scenarios. Human decision making is complex and it is common for such models to have large error terms.

Which method one chooses for an analysis should reflect the scope and purpose of that analysis. Definition of scope can be difficult and arbitrary and can lead to self-defeating policies in the worst cases \cite{Handy_1996}. This study is concerned with the effects of electrification on regional accessibility for road vehicle users. This scope simplifies opportunity selection. It is necessary that a transportation system provide for access between population centers within a region of interest. This study is focused on non-routine regional travel rather than routine local travel as this is where supply infrastructure becomes important. It should be noted that the method is, nevertheless, valid for all travel scales. Said methodology is developed in the following section.