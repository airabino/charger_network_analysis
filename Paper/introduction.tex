\section*{Introduction}

Transportation accessibility is the study of two related phenomena. (1) how structural and individual factors create latent flows or demand. (2) how efficiently transportation systems accommodate said flows. An effective transportation system is an economic and social asset of tremendous value and one which justifies continual investment. Investments in transportation are inherently long-horizon and come with considerable uncertainty. Changes in social, technical, environmental, and other factors can lead to poor returns on investments.

The US and similarly wealthy anglophone nations are unusually car-centric by global standards \cite{PrietoCuriel_2024}. In retrospect, it is apparent that urban development in wealthy nations in the second half of he 20\textsuperscript{th} century led to a self-reinforcing car-dependence. Fully meeting climate goals will necessitate lowering car travel \cite{Milovanoff_2020} but this is necessarily a long-term project. The \gls{ghg} impacts of car travel can be substantially mitigated by large scale vehicle electrification \cite{NA_2021}. Whether or not, and to what extent, such a transition happens in democratic countries will hinge on the fundamentals. Even supply-side policies, such as California's Advanced Clean Cars II, are unlikely to succeed ultimately if it becomes apparent to voters that \glspl{bev} result in dramatically worse access.

Evidence suggests that long itinerary performance is a persistent and important factor in individual car buying decisions. Negative perceptions of \gls{bev} long trip utility on consumer stated preference were found to be impactful in the late 2010s \cite{Skippon_2016, Hardman_2016, Franke_2017, Schmalfuss_2017}. In the intervening time period \gls{bev} ranges and maximum charge rates have markedly increased. Nevertheless, negative perceptions related to long itinerary utility persist for purchasers \cite{Bhat_2022, Paradies_2023, Corradi_2023, Philip_2023} and \gls{bev} range is a significant factor in determining usage share of \glspl{bev} in multi-vehicle household fleets \cite{Chakraborty_2022}.

For long itineraries \glspl{icev} offer greater accessibility compared to \glspl{bev} due to greater maximum ranges and the ubiquitous availability of fueling stations. Fueling stations are widely distributed across urban, suburban, and rural areas, ensuring that drivers have convenient access to refueling points wherever they travel. The charging network is considerably less developed. Disparities between the fueling and DC charging networks result from differences in their underlying economic fundamentals. Gas pumping equipment requires lower up-front costs than DC \gls{evse}, is cheaper to operate \cite{Gamage_2023}, and has been deployed for far longer. Nearly all light-duty \gls{icev} drivers source all of their fuel from public fueling stations regardless of travel behavior. \gls{bev} drivers are expected to, and currently do, source much of their electricity from private AC supply equipment during long dwells \cite{Hardman_2018}. 


The fundamental disparities between \glspl{icev} and \glspl{bev} are well known and well studied. How these manifest as differentials in access is important to understand. This study introduces a novel framework for the assessment of transportation accessibility for long vehicular trips. The methodology measures accessibility by computing optimal-feasible travel routes for \gls{od} pairs using a stochastic routing algorithm subject to vehicle range limitations, supply infrastructure, and driver risk attitudes. This methodology is powertrain agnostic and can be used to directly compare accessibility for different types of vehicles. A case study is presented for the state of California showing a comparison between \glspl{icev} and \glspl{bev} accessibility. The methodology introduced, as well as the open-source code provided in the supplemental information is a valuable tool for planners and policymakers in originating and evaluating \gls{evse} deployment policies.