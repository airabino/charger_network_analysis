\documentclass[11pt]{article}
\usepackage[utf8]{inputenc}
\usepackage[
	letterpaper,
	left = .5in,
	right = .5in,
	top = 1in,
	bottom  = 1in
]{geometry}
\setlength{\columnsep}{.25in}
\usepackage{graphicx}
\usepackage{mathptmx}
\usepackage{float}
\usepackage[cmex10]{amsmath}
\usepackage{amsthm,amssymb}
\usepackage{url}
\urlstyle{same} 
\def\UrlBreaks{\do\/\do-}
\usepackage{breakurl}
\usepackage{fancybox}
\usepackage{breqn}
\usepackage{array}
\usepackage{caption}
\usepackage{subcaption}
%\usepackage{comment}
\usepackage[english]{babel}
\usepackage[acronym,nomain]{glossaries} % list of acronyms
%\usepackage{xurl}
\usepackage{multicol}
\usepackage{multirow}
\usepackage{mathptmx}
\usepackage{float}
\usepackage{lipsum}
%\usepackage{framed}
\usepackage[T1]{fontenc}
\usepackage[pdfpagelabels,pdfusetitle,colorlinks=false,pdfborder={0 0 0}]{hyperref}
%\usepackage{algorithm}
%\usepackage{algpseudocode}
%\usepackage{tabularx}
%\usepackage{wrapfig}
%\usepackage{enumitem}
\usepackage{paralist}

% draw a frame around given text
\newcommand{\framedtext}[1]{%
	\par%
	\noindent\fbox{%
		\parbox{\dimexpr\linewidth-2\fboxsep-2\fboxrule}{#1}%
	}%
}

\renewcommand{\arraystretch}{1.2}

\sloppy
\raggedbottom
\raggedcolumns

\newcolumntype{C}[1]{>{\centering\let\newline\\\arraybackslash\hspace{0pt}}m{#1-2\tabcolsep}}

\usepackage[%
backend=bibtex,     % biber or bibtex
%style=authoryear,    % Alphabeticalsch
style=numeric-comp,  % numerical-compressed
sorting=none,        % no sorting
sortcites=true,      % some other example options ...
block=none,
indexing=false,
citereset=none,
isbn=true,
url=true,
doi=true,           % prints doi
natbib=true,         % if you need natbib functions
]{biblatex}
\addbibresource{./sources/sources.bib,}  % better than \bibliography

\title{A Quantitative Framework for Assessing Long-Trip Transportation Accessibility for Road Vehicles}
\author{Aaron I. Rabinowitz, Vaishnavi Karanam, Radhika Gupta, Gil Tal\\(Alan Jenn?, Tom Bradley?, Susan Handy?, Yueyue Fan?)}
\date{}

\input{gloss}
\makeglossaries

\begin{document}

\maketitle

\section*{Abstract}

A well designed transportation system provides sufficient access from origins to destinations to accommodate household and business demand in an economy.  Increasingly, climate action goals require that more transportation load be shifted to less \gls{ghg} intensive modes among which are \glspl{bev}. While \glspl{bev} use the same roads as \glspl{icev} they draw energy from a separate network of stations which neither as robust as nor coincident to the \gls{icev} fueling network, a consequence of the different current and historical economics of both. Insufficiency and unreliability of public DC \gls{evse} which is primarily used for charging on long itineraries mean that \gls{bev} drivers, depending on vehicle range and risk attitude, may opt for less direct paths with lower charging risk, opt for a more \gls{ghg} intensive mode, or abandon an itinerary. Holistically, the transportation system provides less access to \glspl{bev} for distant pairs. There is also stark disparity between \gls{bev} types. This project develops methods and tools to optimize deployment of future \gls{evse} to mitigate the issue. Methods herein are based on range and charging risk sensitive optimal routing between O/D pairs subject to the locations and usability rates of \gls{evse}. These methods and tools may be used by policymakers to directly evaluate the impact of proposed policies on \gls{bev} transportation access.
\medskip

\begin{multicols}{2}

\section*{Introduction}

Developed economies require the continuous flow of persons and goods in astounding volumes. This dependency justifies investment in the development and maintenance of a transportation sector which is, in its own right, economically significant. It is intuitive that the function of all transportation systems is to enable the actualization of latent activity. It follows that transportation systems should be assessed on that basis and that well designed systems are those which efficiently enable latent flows. Encompassing this paradigm is the field of transportation accessibility. Transportation accessibility is, in short, the study of the related phenomena of how structural and individual factors create latent flows and how transportation systems accommodate them. More efficient built environments minimize the edge traversal costs which exist between demand and supply nodes. More efficient transportation allows for individuals and businesses to access more opportunities for the same cost. In the long term, co-optimization of land-use and transportation is vital to maximizing accessibility.

Transportation accessibility is, inherently, a network problem. Any method which seeks to quantify accessibility must define the nodes and edges which comprise the region in question. In the modern world, all nodes are, to a greater or lesser extent, connected. Scoping an accessibility analysis can be highly determinative of outcome. Researchers and planners have primarily utilized the concept of transportation accessibility as it applies to routine household behavior and local travel \cite{Handy_2020}. From the personal transportation perspective, access is defined as the ease with which individuals can reach the opportunities they desire subject to land-use, transportation infrastructure, temporal availability, and individual preference. Land-use dynamics determine the distribution and demand for amenities such as jobs and services at different locations. Transportation systems determine the edge traversal costs such as travel time, effort, and price which impede flows \cite{Geurs_2004}. Temporal availability of opportunities and transportation modes restrict the utility of opportunities and transportation modes. Finally, individual characteristics such as age, income, and education predict attraction to opportunities  and transportation modes \cite{Miller_2018} effecting their utility.

The US and similar western nations are unusually car-centric by global standards \cite{PrietoCuriel_2024} for personal transportation. The access provided by a road transportation system for \glspl{bev} is different than that for \glspl{icev} due to vehicular and supply network characteristics. Modern \glspl{bev} possess sufficient practical ranges to accomplish much daily of daily travel as shown in Figure \ref{fig:utility_factors} with data from \cite{NHTS_2017, NHTS_2022}. However, for long itineraries \glspl{icev} offer greater accessibility compared to \glspl{bev} due to vehicle ranges and the extensive availability of fueling stations in contrast to DC charging stations. Fueling stations are widely distributed across urban, suburban, and rural areas, ensuring that drivers have convenient access to refueling points wherever they travel. In contrast, the  charging network is less developed and distributed. This infrastructure gap poses challenges for \gls{bev} drivers, especially in remote or less densely populated areas, leading to concerns about range anxiety and limitations on travel options. Inadequate long-trip accessibility for \gls{bev} may result in cancellations or mode switches, often favoring \gls{icev} or air travel.

\begin{figure}[H]
	\centering
	\includegraphics[width = \linewidth]{figs/UF_2017_2022_km.png}
	\caption{Individual vehicle routine travel utility factors as a function of range by powertrain type for \gls{nhts} 2017 and 2022 editions.}
	\label{fig:utility_factors}
\end{figure}

Disparities between the fueling and DC charging networks result from differences in their economic models. Gas pumping equipment requires lower up-front costs than DC \gls{evse}, is cheaper to operate \cite{Gamage_2023}, and has been deployed for far longer. Gas is often sold at low margin with stations making most profit on convenience items. Nearly all light-duty \gls{icev} drivers source all of their fuel from public fueling stations regardless of travel behavior. \gls{bev} drivers are expected to, and currently do, source much of their electricity from AC supply equipment during long dwells, often at private chargers \cite{Hardman_2018}. Thus DC charging stations are subject to higher capital expenditure, lower revenue potential, and less accumulated investment. To overcome these disadvantages, public investments in DC charging infrastructure must be made judiciously. Evaluation methods for potential charging stations should consider their network-wide impact on accessibility considering vehicle types, equipment reliability, and driver risk attitudes.

This study introduces a novel methodology to assess transportation accessibility for long vehicular trips. The methodology measures accessibility by computing optimal-feasible travel routes for \gls{od} pairs using a stochastic routing algorithm subject to vehicle range limitations, supply infrastructure, and driver risk attitudes. This methodology is powertrain agnostic and can be used to directly compare accessibility for vehicles with different ranges and which rely on different supply networks. Additionally, a case study is presented for the state of California showing a comparison between \glspl{icev} and \glspl{bev} access for important \gls{od} pairs within the state. The methodology introduced, as well as the open-source code provided in the supplemental information is a valuable tool for planners and policymakers in originating and evaluating \gls{evse} deployment policies.

\section*{Transportation Accessibility}

Transportation accessibility has been studied as a tool for urban and regional planners since the middle of the 20\textsuperscript{th} century. Accessibility derives from the theory of population migration proposed by Ravenstein in 1885 \cite{Ravenstein_1885}. The movement of populations over a given time-scale can be analogized to DC power flow. In this analogy push and pull factors determine the "voltage" separation, traversal difficulty is the "resistance" and the resulting "current" is the flow. The field of transportation accessibility uses this framework to study regional efficiency considering both voltage and resistance simultaneously. The accepted definition of transportation accessibility is the ease with which individuals can access relevant opportunities subject to the transportation system in the relevant area. Thus, accessibility is a framework which encompasses voltage factors such as land use and temporal availability, resistance factors such as transportation system design, and universal factors such as personal preference \cite{Geurs_2004}. Literature provides four essential frameworks for computing access as surveyed in \cite{Handy_1997, Kwan_1998, Geurs_2004, Miller_2018, Handy_2020} and discussed below.

Much of the difference between methodologies is in the selection of opportunities. Individuals are assumed to need or desire location-specific opportunities such as employment and physical retail. However, there may be several near-equivalents for any given opportunity type. The simplest methods for selecting opportunities are based on nearest proximity \cite{Wachs_1973, Vickerman_1974}. Proximity methods consider that a person has a level of access to a given need as determined by that person's proximity to the closest relevant opportunity. These methods do not account for heterogeneity within an opportunity category nor for the benefits of redundancy within an opportunity category. The inverse are isocost methods wherein a person is said to have access as determined by the number of opportunities available within a given isocost polygon. This method has the drawback of not considering the differences in edge traversal cost for \gls{od} pairs within the isocost region. These methods have been used widely \cite{Easa_1993} due to their computational lightness and form the basis for modern big-data methods such as the US DOE's Mobility Energy Productivity metric \cite{Hou_2019}.

Proximity and isocost methods are easy to compute because they treat redundancy arbitrarily. In practice, equivalent and near-equivalent opportunities compete with one-another if sufficiently proximate or if the paths required to reach them overlap \cite{Stouffer_1940}. Gravity/entropy methods \cite{Noulas_2012, Jung_2008} address this shortcoming. These methods are so called as they concern the cumulative effect of opportunities for a given origin on the basis of demand over proximity (gravity) or information content (entropy). Such methods were first formalized into a quantitative framework in 1959 \cite{Hansen_1959} as a generalization of previous methodology for quantifying the efficiency of urban land use. Gravity/entropy methods define accessibility as the intensity of the possibility for interaction. Implicit in the formulation of gravity/entropy methods is that every opportunity has some effect on every individual, even if negligible, and the effect of any one opportunity is determined by its network position.

Proximity and gravity/entropy methods rely on the assumption that traversal cost is the primary factor determining individuals decision to select one opportunity from among a set of similar entities. While this is certainly true if the difference in traversal cost is large enough it is not, altogether, obvious what the threshold of disambiguation is for a given individual. Thus, researchers have proposed to use Discrete Choice Modeling \cite{Ben_Akiva_1985} to explain revealed choices wherein ease-of-access is one of several possible factors in determining the utility of a given opportunity for an individual \cite{Cevero_1995, Shen_1998, Karst_2003}.

There are, thus, a variety of methods which can be used to quantify the accessibility of a given region with varying computational and data requirements. The relationship between land-use, transportation, and demography is circular rather than linear. Which method one chooses for an analysis reflects the scope and purpose of that analysis. Definition of scope can be difficult and can lead to self-defeating policies \cite{Handy_1996}. This study is concerned with the effects of electrification on long-trip accessibility for road vehicle users. This scope simplifies opportunity selection. It is necessary that a transportation system provide for access between large population centers within a region of interest and to those in adjacent regions.  This study is focused on non-routine regional travel rather than routine local travel as this is where supply infrastructure becomes important. It should be noted that the method is valid for all travel scales. Methodology is developed in the following section.

\section*{Methods}

The focus of this study is long-trip accessibility for road vehicles. The metric proposed is of the Proximity type and reflects the ease with which a given individual, driving a given vehicle, can access the important locations in a region from one-another. The metric is titled Road-trip Accessibility. It is assumed that travelers, when considering non-routine long trips have a specific destination in mind and would not consider nearly equivalent and/or equidistant locations as fungible. Rather, having determined to travel to a given location the traveler will then select a mode. Road-trip Accessibility is computed by calculating the mode and person specific lowest-cost-paths for all pairs of selected locations within the region and taking the average. Example costs which might be used are travel time and travel price. This paper will focus, exclusively, on travel time. The manner in which lowest-cost-paths are calculated is critical in arriving at a representative result. This study calls for a specific method of stochastic route optimization and evaluation which will be detailed in the subsections below.

A transportation accessibility metric should have land-use, transportation, temporal, and individual components \cite{Karst_2003}. Road-trip Accessibility contains all of these components. The land-use within a region has two principle effects on Road-trip Accessibility. First, for a multi-city region, peripheral cities should experience worse regional access than central cities. Second, geographically large and/or sparse regions should experience worse overall accessibility than compact regions. The transportation infrastructure determines the efficacy of various modes. Where only vehicular travel is concerned, the mode choice is reduced to vehicle choice. Vehicle specific infrastructure is not necessarily the same for all vehicles of a given fuel type but is necessarily different between fuel types creating effectively separate modes. The temporal component of Road-trip Accessibility arises from the schedules of public transportation services and road traffic patterns. Finally the individual component is the traveler risk attitude and route cost weights.

\subsection*{Road-trip Accessibility Metric Definition}

For region $R$ with a set of $N$ selected locations $O = \{O_1, O_2, \dots, O_N\}$ and a corresponding set of importance weights $W = \{W_0, W_1, \dots, W_N\}$, the Road-trip Accessibility metric $A_{rt}$ is

\begin{equation}
	A_{rt} = \frac{1}{N^2}\sum_{i = 0}^{N} \sum_{j = 0 }^{N} W_iW_jC(O_i, O_j) \label{eq:a_rt}
\end{equation}

\noindent where $C(O_i, O_j) $ is the cost of the optimal path from $O_i$ to $O_j$. The specific accessibility metric $A_{rt,i}$ can be computed for region $R$ and origin $i$ as

\begin{equation}
	A_{rt,i} = \frac{1}{N}\sum_{j = 0 }^{N} W_iW_jC(O_i, O_j) \label{eq:a_rt_i}
\end{equation}

for any origin $O_i \in O$. Road-trip Accessibility is a weighted mean of costs to access important locations from either a specific location or for all-pairs. The all-pairs implementation \eqref{eq:a_rt} is titled General Road-trip Accessibility while the single-origin implementation \eqref{eq:a_rt_i} is titled Specific Road-trip Accessibility. A given score applies to a given region, travel mode, and individual person. Comparisons may be made between different combinations by computing scores for each combination.

\subsection*{Computing the Road-trip Accessibility Metric}

Powered vehicles are range-limited due to the finite capacity of their \glspl{ess}. In order to traverse an \gls{od} arc whose energy requirement is greater than a given limit, a vehicle must stop at a supply station. In practical terms, the limit is defined by the vehicle's \gls{ess} capacity, starting \gls{soc}, desired finishing \gls{soc}, and the driver's low \gls{soc} tolerance. Because supply events add time to a trip, they will usually be avoided on short trips. For sufficiently long trips, where at least one supply event is necessary, computing the lowest-cost-path requires considering the time added by supply events and the locations of supply stations. These are, usually, not considered as, for \glspl{icev}, supply events are brief and supply stations are ubiquitous in most areas of the developed world. However, for \glspl{bev}, where supply events are lengthy and where DC supply stations are comparatively rare, ignoring supply events when computing a lowest-cost-path carries non-negligible risk. For this reason, dedicated \gls{bev} routing services compute routes considering supply events [SOURCE  - a better route planner].

A further complication is that the driver will have limited and uncertain information with which to evaluate alternatives at the beginning of the trip. Variables such as traffic, temporary road closures, supply equipment non-functionality, and queues at supply stations cannot be precisely known at the start of a trip. The ability to estimate delays and the severity of delays vary among the previously mentioned events. As the driver progresses, certainty may increase but factors such as charger availability may change up to the instant that the driver arrives at the charger. As such, the driver will have to plan to optimize the expected result. This is replicated via stochastic optimization.

\subsubsection*{Stochastic Optimal Routing}

The purpose of stochastic optimal routing is to find the expected lowest-cost-paths from a given origin $i \in V$ to a set of destinations $D \in V$ on a graph $G = \{V, E\}$. The output is tree $P$ containing the optimal-feasible paths from the origin to the selected destinations. The objective of routing between nodes $i$ and $j\in D$ is

\begin{equation}
	\min_{U \in \overline({U}_{i,j})}\ \mathbb{E}[J(S_0, U)]
\end{equation}

where

\begin{equation}
	J(U) = \sum_{k = 0}^M \Phi_k(S_0, U)
\end{equation}

s.t.

\begin{gather}	
	b^k_l \leq \mathbb{E}\left[\int_0^t \Phi_k(S_0, U)dt\right] \leq b^k_u\\
	\mathbb{E}\left[\int_0^T \Phi_k(S_0, U)dt\right] \geq b^k_f\\
	 t \in [0, T]\quad k = 1, 2, \dots, M
\end{gather}

\noindent where $T$ is the final value of time for a route, $S$ is the state vector, $S_0$ is the initial values of the states, $U$ is a control (route) between $i$ and $j$, $\overline{U}_{i,j}$ is the set of possible routes between $i$, and $j$, $\Phi$ is the set of cost functions, $b^k_l$ and $b^k_u$ are the upper and lower bounds for state $k$ respectively, and $b^k_f$ is the final state minimum value for state $k$. $\mathbb{E}$ denotes an expectation. State vector $S$ is initialized and stored as vectors containing $N$ discreet variables. A distribution $D$ for a the state vector at any node and time-step can be computed from a histogram of the values. States can be changed at nodes and edges. Routes are considered feasible if state expectations remain within set bounds. Comparison between routes is performed using cost expectation. Routing depends on vehicle, infrastructural, and individual models.


\subsubsection*{Vehicle Model}

Vehicles effect routing due to their range limits and supply methods. The vehicle model used herein is highly simplified due to the inexact nature of the problem. Vehicles serve to store energy and consume energy at a constant rate per unit distance driven. More exact information on road conditions, traffic conditions, and atmospheric conditions among others can be used to compute edge-specific efficiencies. Vehicular parameters are listed in Table \ref{tab:param_veh}.

\begin{table}[H]
	\centering
	\caption{Vehicle Parameters for Routing}
	\label{tab:param_veh}
	\begin{tabular}{|C{\linewidth*3/8}|C{\linewidth*3/8}|C{\linewidth*2/8}|}
		\hline Parameter & Description & Unit \\
		\hline \gls{ess} Capacity & Accessible energy storage capacity & [kWh] \\
		\hline Energy Consumption & Energy required to move the vehicle & [kJ/km] \\
		\hline Supply Rate & \gls{ess} maximum energy addition rate & [kW] \\
		\hline Supply Limits & \gls{soc} bounds for supply & [-] \\
		\hline
	\end{tabular}
\end{table}

\subsubsection*{Supply Infrastructure Network Model}

Supply networks effect routing both in structure and in the characteristics of individual stations. In this paper, the supply network refers to all stations which the vehicle can utilize rather than the industry definition which covers stations operated by a single \gls{cpo}. Networks ultimately consist of individual supply ports (chargers or fuel pumps) and serve large and geographically distributed demand. A network consisting of more than one port can develop redundancy either by concentrating ports in a single confined space "in-station" or via a more evenly distributed approach "between-station". Network redundancy also varies by location with "thinner" and "thicker" coverage areas.

Vehicle optimal routing serves to find the lowest-cost-path between two points on a road atlas atlas. A road atlas is a graph with a large number of nodes and a low edge-to-node ratio. When considering a multi-leg itinerary, such as when solving the Traveling Salesman Problem or Vehicle Routing Problem, the atlas need not be stored in entirety. Only the locations of and relationships between relevant nodes are necessary. The graph formed from the relevant nodes and their relationships is a reduced subgraph with a low number of nodes and relatively high edges-per-node and cycles-per-node ratios. A reduced subgraph is defined as follows. For a graph $G = \{V, E\}$ where $V$ is the set of nodes and $E$ is the set of edges, a reduced subgraph $\hat{G} = \{\hat{V}, \hat{E}\}$ can be computed where $\hat{V} \subseteq V$ and $\hat{E}$ is the set of shortest-paths between all nodes in $\hat{V}$. An example of a reduced subgraph is shown in Figure \ref{fig:reduced_subgraph}.

\begin{figure}[H]
	\centering
	\begin{subfigure}[t]{.5\linewidth}
		\centering\captionsetup{width = .8\linewidth}
		\includegraphics[width = \linewidth]{figs/full_graph.png}
%		\caption{Locations and routes between selected nodes on atlas}
	\end{subfigure}%
	\begin{subfigure}[t]{.5\linewidth}
		\centering\captionsetup{width = .8\linewidth}
		\includegraphics[width = \linewidth]{figs/reduced_graph.png}
%		\caption{Reduced graph containing locations of and relationships between selected nodes}
	\end{subfigure}
	\caption{Example original graph (a) containing locations and an atlas and reduced subgraph (b) containing locations and arcs.}
%	The locations selected are key destinations in the state or on its border. Logistical optimization requires, first computing arcs between \gls{od} pairs as in the right pane based on the atlas in the left pane.}
	\label{fig:reduced_subgraph}
\end{figure}

For a given vehicle type, the \gls{sng} is the reduced subgraph containing the trip origin, trip destination, and all supply stations reasonably likely to be utilized. Alternatively the \gls{sng} can contain a set of origins and destinations in order to enable further reduction for logistical optimization purposes. The \gls{sng} is the graph on which long vehicle trips should be optimized if supply events are expensive and/or constraining. In general, because the road atlas will have a low cycle-to-node ratio, the Bellman algorithm will be faster while the Bellman-Ford algorithm (specifically the SPFA variant) will be faster for the \gls{sng} with its high cycle-to-node ratio. The relevant \gls{sng} for \glspl{icev} and \gls{bev} are neither equivalent nor isomorphic. Different vehicles within a given powertrain type may also have different \glspl{sng} but this is far more common for \glspl{ev} than \glspl{icev}. The \gls{sng} informs routing by providing a set of possible paths between origins and destinations. The structure of a network may make for a greater or lesser number of available paths depending on the number and location of stations.

Supply station characteristics significantly impact the expected cost of utilizing a given station. Supply stations are defined by the number of ports, the reliability of ports, the maximum supply rate of ports, the rate of user arrivals, and the durations of supply events. In combination, these factors determine the likelihood of a port being usable and available as well as the likely duration of queue if no port is usable and available. Supply station parameters are listed in Table \ref{tab:param_supply}.

\begin{table}[H]
	\centering
	\caption{Supply Station Parameters for Routing}
	\label{tab:param_supply}
	\begin{tabular}{|C{\linewidth*2/8}|C{\linewidth*3/8}|C{\linewidth*3/8}|}
		\hline Parameter & Description & Unit \\
		\hline Arrivals Ratio & Arrival rate at station per port and unit of time & [cars/port/hour] \\
		\hline Supply Rate & Maximum rate of energy supply & [kW] \\
		\hline Supply Energy & Expected energy supplied to vehicles & [kWh] \\
		\hline Ports & Number of chargers/pumps at a station which can be used simultaneously & [-] \\
		\hline Reliability & Percentage of the time that a given pump will be usable & [-] \\ 
		\hline
	\end{tabular}
\end{table}

Unfortunately, with the exception of the number of ports and the maximum supply rate, little of this information is available before arrival [SOURCE]. As a result, arrivals ratio, supply energy, and reliability must be estimated. To reflect this uncertainty, distributions are used for arrivals ratio and supply energy. Information on ports is taken from \gls{afdc} \cite{afdc_2023}, and information on chargers is taken from \cite{Rempel_2023}. The delays which result from DC charging station parameters are modeled as M/M/s queues and are demonstrated in \ref{fig:expected_delay}.

\begin{figure}[H]
	\centering
	\includegraphics[width = \linewidth]{figs/expected_delay.png}
	\caption{Expected queuing time at charging stations. Low arrival rate is 10 - 60 minutes between arrivals. High arrival rate is 1 - 10 minutes between arrivals. All charge rates are 80 kW and charge events are 60 $\pm$ 15 kWh.}
	\label{fig:expected_delay}
\end{figure}

The mean arrivals rate is not known precisely, nor is the mean service rate so each is sampled from a distribution. In Figure \ref{fig:expected_delay} the columns show different distributions of arrivals rate and the rows show different levels of port redundancy. It can be observed that each factor is substantial in determining expected queue length and that their effects are additive. Without better information, a driver will have to make an educated guess as to how busy a station is likely to be and how many of its listed chargers will be functional. Getting either factor wrong could have serious consequences if the driver does not have the option to easily reach a different station where the driver may have no better information regardless. Reflecting the lack of information this study uses very broad assumptions for arrivals rate distribution and charge energy distribution. Arrivals rate is assumed to be a ratio of vehicles per hour per charger normally distributed as $\mathcal{N}(1, 0.5)$. Charge energy delivered is assumed to be normally distributed as $\mathcal{N}(60, 15)$ in kWh. \gls{icev} supply infrastructure is assumed to be ubiquitous with massive redundancy within and between stations such that supply requires neither deviation nor delay.

\subsubsection*{Drivers}

Given the same physical circumstances, different drivers will evaluate route costs differently. In a basic sense, drivers will weight several factors such as time, money, distance, and complexity differently. Where any important factor is not known precisely drivers will consider a range of outcomes and decide based on an expectation. Driver risk attitude concerns what range of outcomes will be used to compute expected cost. Risk attitude is modeled using a superquantile risk function defined as

\begin{equation}
	S(D, p_0, p_1) = \frac{1}{p_1 - p_0}\int_{p_0}^{p_1}Q(D, \alpha)\ d\alpha \label{eq:superquantile}
\end{equation}

\noindent where $D$ is a distribution, $p_0$ and $p_1$ are the boundaries of the range of probabilities considered in the expectation, and $Q$ is the quantile function of $D$. The superquantile is, thus, the mean value of a distributed quantity within a range of probability. $S(D, 0, 1)$ reduces to the mean of $D$. Drivers with an aggressive risk attitude will consider a low range of probabilities. Drivers with a neutral attitude will consider a central range of probabilities. Drivers with a cautious attitude will consider a high range of probabilities. Driver parameters are listed in Table \ref{tab:param_driver}.

\begin{table}[H]
	\centering
	\caption{Supply Station Parameters for Routing}
	\label{tab:param_driver}
	\begin{tabular}{|C{\linewidth*3/8}|C{\linewidth*3/8}|C{\linewidth*2/8}|}
		\hline Parameter & Description & Unit \\
		\hline Route Cost Weights $W$ & Set of multipliers for route costs to be used in computation of weighted sum & [-] \\
		\hline Probability Range $(p_0, p_1)$ & Range of probabilities for superquantile function & [-] \\
		\hline
	\end{tabular}
\end{table}

The driver model serves to bias the routing by selecting a subset of information to use in optimization. As such it reflects individual perception. The results of the routing can be interpreted through the same bias, a different bias, or no bias. Thus Road-trip accessibility can be as perceived (with bias) ot as expected (without bias). In this study, evaluation will be conducted on the basis of expectation.

\subsection*{Randomly Generated Example}

The Road-trip accessibility method and metric is demonstrated using a vehicle on a randomly generated \gls{sng}. In this example, a \gls{sng} containing 15 locations and 85 stations distributed across a 1000 km square is generated with randomized node locations. Edges exist between all nodes and are assigned Pythagorean distances. All edges are assumed to be traversed at 105 kmh. The origin is selected as Location 0 located in the lower-right corner with all other locations as destinations and a Specific Road-trip Accessibility is computed for Location 0. The vehicle used has a relatively short range of 262 km. The charging stations have between 1 and 5 chargers. Figure \ref{fig:perceived_srta_random} shows the times to travel to each destination as well as the path taken along the \gls{sng} under four different scenarios.

\begin{figure}[H]
	\centering
	\begin{subfigure}[t]{.5\linewidth}
		\centering\captionsetup{width = .8\linewidth}
		\includegraphics[width = \linewidth]{figs/random_example_high_reliability_aggressive.png}
		\caption{High reliability, aggressive}
	\end{subfigure}%
	\begin{subfigure}[t]{.5\linewidth}
		\centering\captionsetup{width = .8\linewidth}
		\includegraphics[width = \linewidth]{figs/random_example_high_reliability_cautious.png}
		\caption{High reliability, cautious}
	\end{subfigure}
	\begin{subfigure}[t]{.5\linewidth}
		\centering\captionsetup{width = .8\linewidth}
		\includegraphics[width = \linewidth]{figs/random_example_low_reliability_aggressive.png}
		\caption{Low reliability, aggressive}
	\end{subfigure}%
	\begin{subfigure}[t]{.5\linewidth}
		\centering\captionsetup{width = .8\linewidth}
		\includegraphics[width = \linewidth]{figs/random_example_low_reliability_cautious.png}
		\caption{Low reliability, cautious}
	\end{subfigure}
	\caption{Driver's perception of Specific Road-trip Accessibility for different drivers with different levels of charger reliability.}
	\label{fig:perceived_srta_random}
\end{figure}

The scenarios presented in Figure \ref{fig:perceived_srta_random} consider different levels of equipment reliability (75\% and 95\%) and different driver risk-attitudes (aggressive - $p_0 = 0,\ p_1 = .1$ and cautious - $p_0 = .9,\ p_1 = 1$). Specifically, the results consider Specific Road-Trip Accessibility as perceived by the driver. The aggressive driver is only concerned with the best 10\% of outcomes where the cautious driver is only concerned with the worst 10\% of outcomes. The differences in perceived costs-to-travel are quite stark between the aggressive and cautious driver in both cases but this difference is larger when reliability is low. In other words, the effects are additive from a perceived cost perspective. A manifestation of this gap is the different routes taken in the different scenarios. Perception is biased and not necessarily the best basis to evaluate costs-to-travel. The neutral expectations ($p_0 = 0,\ p_1 = 1$) of the routes taken by the drivers are shown in Figure \ref{fig:perceived_srta_random_n}.

\begin{figure}[H]
	\centering
	\begin{subfigure}[t]{.5\linewidth}
		\centering\captionsetup{width = .8\linewidth}
		\includegraphics[width = \linewidth]{figs/random_example_high_reliability_aggressive_n.png}
		\caption{High reliability, aggressive}
	\end{subfigure}%
	\begin{subfigure}[t]{.5\linewidth}
		\centering\captionsetup{width = .8\linewidth}
		\includegraphics[width = \linewidth]{figs/random_example_high_reliability_cautious_n.png}
		\caption{High reliability, cautious}
	\end{subfigure}
	\begin{subfigure}[t]{.5\linewidth}
		\centering\captionsetup{width = .8\linewidth}
		\includegraphics[width = \linewidth]{figs/random_example_low_reliability_aggressive_n.png}
		\caption{Low reliability, aggressive}
	\end{subfigure}%
	\begin{subfigure}[t]{.5\linewidth}
		\centering\captionsetup{width = .8\linewidth}
		\includegraphics[width = \linewidth]{figs/random_example_low_reliability_cautious_n.png}
		\caption{Low reliability, cautious}
	\end{subfigure}
	\caption{Neutral expectation of Specific Road-trip Accessibility for different drivers with different levels of charger reliability.}
	\label{fig:perceived_srta_random_n}
\end{figure}

The neutral expectations tell a different story. For the different levels of reliability, neutral expectations of costs-to-travel are similar for the routes selected by the aggressive and cautious drivers. The differences between biased and neutral perception are easy to understand. Both drivers selected routes based on a subset of the information and these routes are non-optimal when all information is considered. Drivers may adjust their priors over time but policy can only control the fundamentals. Thus, it is recommended to consider bias in route planning but neutral expectation in evaluation.

The composition of the routes taken by the drivers in the randomly generated example demonstrate an interesting dynamic as seen in Table \ref{tab:distances_redundancy}.

\begin{table}[H]
	\centering
	\caption{Average route distances and chargers per station utilized for example scenarios}
	\label{tab:distances_redundancy}
	\begin{tabular}{|C{.25\linewidth}|C{.25\linewidth}|C{.25\linewidth}|C{.25\linewidth}|}
		\hline Reliability & Risk-Attitude & Average Route Distance [km] & Chargers per Station Utilized [-] \\
		\hline High & Aggressive & 558.827 & 3.871 \\
		\hline High & Cautious & 578.505 & 4.290 \\
		\hline Low & Aggressive & 575.947 & 4.063 \\
		\hline Low & Cautious & 574.953 & 3.750 \\
		\hline
	\end{tabular}
\end{table}

When reliability is high, the aggressive driver opts for a more direct path where the cautious driver favors a path with higher charger redundancy in-station. However, when reliability is low the reverse happens. This is because, in the low reliability scenario, completely failed stations come into play and the cautious driver tends to stick to areas with higher redundancy between-station even if this means accepting perceived longer queues. The aggressive driver, ignoring the worst case scenario of being stranded, opts for the lower expected queuing time to be found in high redundancy stations.

\section*{California Case Study}

The randomly generated example is informative but does not reflect any actual \gls{sng}. In order to see effects on a more representative basis a case study is performed on the state of California using information on the states DC EV \gls{sng} with modes of common \glspl{bev} which enjoy different levels of access.

\subsection*{Background}

The state of California is geographically large and contains major population centers distributed across the state. Major road transportation corridors form connections between the state's population centers and with population centers in adjacent states. This case study concerns the long-trip accessibility of California's road transportation network. For the purposes of analysis, 15 important cities in California and adjacent states were selected. Non-California locations are represented by the most-likely departure locations at the California state line. Because the state-line locations are midpoints the required arrival \gls{soc} at these points is set at 50\% where it is set at 20\% for all others. These locations are enumerated in Table \ref{tab:locations}.

\begin{table}[H]
	\centering
	\caption{Locations Considered for Long Trip Accessibility}
	\label{tab:locations}
	\begin{tabular}{|C{\linewidth * 1 / 3}|C{\linewidth * 2 / 3}|}
		\hline Index & Location \\
		\hline 0 & Crescent City \\
		\hline 1 & Yreka \\
		\hline 2 & Redding \\
		\hline 3 & Chico \\
		\hline 4 & Reno (State Line) \\
		\hline 5 & Sacramento \\
		\hline 6 & Stockton \\
		\hline 7 & San Francisco \\
		\hline 8 & San Jose \\
		\hline 9 & Fresno \\
		\hline 10 & Las Vegas (State Line) \\
		\hline 11 & Bakersfield \\
		\hline 12 & Los Angeles \\
		\hline 13 & Phoenix (State Line) \\
		\hline 14 & San Diego \\
		\hline
	\end{tabular}
\end{table}

For long trips, \glspl{bev} will rely on DC charging stations. The locations of all DC charging stations in California are available from AFDC \cite{afdc_2023}. California's DC charging stations include proprietary (vehicle \gls{oem} owned and operated) stations such as Tesla Superchargers and the Rivian Adventure network as well as non-proprietary stations such as those operated by ChargePoint, Electrify America, eVgo, and others. The selected locations and DC charging stations are mapped in Figure \ref{fig:california_atlas}.

\begin{figure}[H]
	\centering
	\includegraphics[width = \linewidth]{figs/California_Places_Chargers.png}
	\caption{Selected locations and DC charging stations for California case study}
	\label{fig:california_atlas}
\end{figure}

The proprietary and non-proprietary networks in California are neither equivalent nor isomorphic. There are 403 Tesla and 78 Rivian DC charging stations in the state as compared to 1,425 non-proprietary DC charging stations. The non-Tesla networks overwhelmingly use combination CCS/ChaDeMo chargers as defined by SAE J1772 \cite{SAE_J1772} which reflect the ports on the overwhelming number of non-Tesla \glspl{bev}. By contrast, Tesla chargers and vehicles use the NACS standard as defined by SAE J3400 \cite{SAE_J3400}. The Tesla and non-Tesla systems are increasingly interoperable with the aid of adapters but should be considered separately in the present. Tesla drivers use Tesla DC chargers almost exclusively \cite{Visaria_2022} and CCS/ChaDeMo to NACS adapters are more common than their counterparts. The Rivian Adventure network is technically interoperable with other J1772 vehicles but is set aside for the exclusive use of Rivian vehicles. The purpose of the Rivian Adventure network serves to allow for Rivian vehicles to charge in remote locations and is not intended to be relied upon exclusively.

The difference between the Tesla DC charging network and the non-Tesla networks extends from function to form. Built out as an investment to entice sales of Tesla vehicles and, until recently, exclusive to them, the Tesla network is technically superior with higher maximum charging rates and more reliable chargers \cite{Rempel_2023, Kozumplik_2022}. Non-proprietary networks have, so far, been utilization and subsidy driven \cite{Gamage_2023} and have responded to incentives which encourage widely distributed stations with few chargers per station. A stark contrast is seen when examining the ratio of chargers to stations. In California there are 403 Tesla DC charging stations with a total of 6,277 DC chargers for an average of 15.6 chargers per station. Among non-proprietary networks there are a total of 1,425 stations with 3,667 chargers for an average of 2.6 per station. The distributions of chargers per station for Tesla and non-proprietary networks are shown in Figure \ref{fig:network_histograms}. 

\begin{figure}[H]
	\centering
	\includegraphics[width = \linewidth]{figs/California_Charger_Network_Survival_Functions.png}
	\caption{Survival functions for charger networks in California. Top panel shows survival function for chargers per station. Bottom panel shows survival function for mean distance to three nearest stations from a given station.}
	\label{fig:network_histograms}
\end{figure}

The Tesla DC charging network develops redundancy primarily in-station where the non-proprietary networks develop redundancy primarily between-station. Non-Tesla chargers are also more likely to be sighted in urban areas suggesting a desire to capture local as well as corridor travel demand. Tesla stations are more often sighted along travel corridors suggesting a focus on enabling long distance travel. In more remote parts of California the proprietary networks nearly match the non-proprietary networks in between-station redundancy.

California \glspl{icev} utilize a third and completely separate network of supply stations. There are estimated to be over 8,000 gasoline stations in California \cite{CEC_2022} and these are widely and proportionally distributed. Because no public database for the locations of gasoline stations in the state exists, and due to their ubiquity it is assumed in this study that \gls{icev} driver optimal paths will not be effected by fueling station availability and that such stations will be available wherever needed. For this reason, \glspl{icev} are assumed to take the "direct" path between cities where \glspl{bev} need to find optimal paths on their \glspl{sng}.

\subsection*{Example}

Consider, as an example, the following four scenarios for a driver based out of Fresno:


\begin{compactenum}
	\item Risk neutral driver using a generic \gls{icev} with an \gls{ess} capacity of 550 kWh and an efficiency of 2700 kJ/km.
	\item Risk neutral driver using a Tesla Model 3 with an 80 kWh battery and efficiency of 536.4 kJ/km capable of charging at a max rate of 170 kW. The driver uses the Tesla DC station network exclusively and this network has high charger up-time (97\%). The driver only charges to 80\% \gls{soc} for DC events.
	\item Risk-cautious driver using a Chevrolet Bolt EV with an 65 kWh battery and efficiency of 626.5 kJ/km capable of charging at a max rate of 55 kW. The driver uses various non-proprietary networks and these networks have low charger up-time (75\%). The driver only charges to 80\% \gls{soc} for DC events.
	\item Risk-aggressive driver using a Chevrolet Bolt EV with an 65 kWh battery and efficiency of 626.5 kJ/km capable of charging at a max rate of 55 kW. The driver uses various non-proprietary networks and these networks have low charger up-time (75\%). The driver only charges to 80\% \gls{soc} for DC events.
\end{compactenum}

Each of these drivers will have a different experience of California's road transportation system for long trips. The neutral-expectations of time to reach each of the selected locations in the case study are provided in Table \ref{tab:scenarios} (driver is required to arrive at each destination with  at least 20\% \gls{soc}).

\begin{table}[H]
	\centering
	\caption{Neutral expectation of hours to locations from Fresno for example scenarios.}
	\label{tab:scenarios}
	\begin{tabular}{|C{.17\linewidth}|C{\linewidth * 1 / 5}|C{\linewidth * 1 / 5}|C{\linewidth * 1 / 5}|C{.23\linewidth}|}
		\hline Index & \gls{icev} & Model 3 Neutral & Bolt Cautious & Bolt Aggressive \\
		\hline 0 & 8.79 & 9.28 & 15.58 & 13.66 \\
		\hline 1 & 6.73 & 7.07 & 13.31 & 10.04 \\
		\hline 2 & 5.28 & 5.60 & 8.78 & 8.35 \\
		\hline 3 & 4.40 & 4.54 & 7.54 & 5.96 \\
		\hline 4 & 4.63 & 4.74 & 7.85 & 6.34 \\
		\hline 5 & 2.79 & 2.82 & 4.01 & 4.17 \\
		\hline 6 & 2.09 & 2.09 & 2.09 & 2.09 \\
		\hline 7 & 3.08 & 3.18 & 4.34 & 4.50 \\
		\hline 8 & 2.71 & 2.71 & 2.71 & 2.71 \\
		\hline 9 & 0.00 & 0.00 & 0.00 & 0.00 \\
		\hline 10 & 5.86 & 6.13 & 8.88 & 9.05 \\
		\hline 11 & 1.64 & 1.64 & 1.64 & 1.64 \\
		\hline 12 & 3.32 & 3.32 & 4.72 & 4.73 \\
		\hline 13 & 6.80 & 7.31 & 11.21 & 10.50 \\
		\hline 14 & 5.29 & 5.47 & 8.41 & 8.32 \\
		\hline
	\end{tabular}
\end{table}

The example drivers have very different experiences and perceived experiences. While the \gls{icev} driver can take the "direct" path (not deviating to find a station), the \gls{bev} drivers have to deviate from the "direct" path and require substantial time to charge. However, there is quite significant disparity within the set of proposed \gls{bev} drivers. For destinations which are within full-charge range, there is no difference between any of the drivers experiences. However, due to the lower range, lower max charge rate, and lower infrastructure reliability for the Bolt, the differentials between it and the Model 3 can be tremendous, especially for proximate and remote locations. It is worth noting, also, that the \gls{icev} and Model 3 are often able to take more direct routes. Table \ref{tab:scenarios_nc} shows the optimal route times with charging/fueling times removed.

\begin{table}[H]
	\centering
	\caption{Neutral expectation of hours to locations from Fresno for example scenarios without charging/fueling time.}
	\label{tab:scenarios_nc}
	\begin{tabular}{|C{.17\linewidth}|C{\linewidth * 1 / 5}|C{\linewidth * 1 / 5}|C{\linewidth * 1 / 5}|C{.23\linewidth}|}
		\hline Index & \gls{icev} & Model 3 Neutral & Bolt Cautious & Bolt Aggressive \\
		\hline 0 & 8.76 & 8.80 & 9.58 & 8.76 \\
		\hline 1 & 6.71 & 6.75 & 7.31 & 6.72 \\
		\hline 2 & 5.26 & 5.34 & 5.78 & 5.33 \\
		\hline 3 & 4.39 & 4.42 & 4.54 & 4.39 \\
		\hline 4 & 4.63 & 4.63 & 4.85 & 4.67 \\
		\hline 5 & 2.78 & 2.82 & 2.81 & 2.81 \\
		\hline 6 & 2.09 & 2.09 & 2.09 & 2.09 \\
		\hline 7 & 3.07 & 3.07 & 3.07 & 3.07 \\
		\hline 8 & 2.71 & 2.71 & 2.71 & 2.71 \\
		\hline 9 & 0.00 & 0.00 & 0.00 & 0.00 \\
		\hline 10 & 5.86 & 5.89 & 5.88 & 5.86 \\
		\hline 11 & 1.64 & 1.64 & 1.64 & 1.64 \\
		\hline 12 & 3.32 & 3.32 & 3.32 & 3.32 \\
		\hline 13 & 6.79 & 6.93 & 6.81 & 7.14 \\
		\hline 14 & 5.28 & 5.28 & 5.41 & 5.28 \\
		\hline
	\end{tabular}
\end{table}

The differences in total expected travel time between the \gls{icev} and Model 3 are relatively small and in the range of 5-10\% of travel time, a testament to Tesla's DC station network. Some would argue that this difference is unimportant as Tesla drivers can charge their vehicles while stopping for meals or during other natural breaks. This logic has been used to show that \glspl{bev} may approach convenience parity with \glspl{icev} in good circumstances \cite{Dixon_2020}. However, where such breaks are optional for \gls{icev} drivers, they are mandatory for \gls{bev} drivers and must be taken at specific points throughout the trip to coincide with charging. The loss of optionality must be accounted an inconvenience even if breaks would be taken in any event.

\subsection*{Experiment}

In order to gain an understanding of the effects of vehicular, infrastructural, and individual characteristics on transportation accessibility for long trips in California a designed experiment was conducted on the parameters in Table \ref{tab:exp_parameters}.

\begin{table}[H]
	\centering
	\caption{Parameters and levels for designed experiment}
	\label{tab:exp_parameters}
	\begin{tabular}{|C{.5\linewidth}|C{.5\linewidth}|}
		\hline Parameter & Levels \\
		\hline Charger Network & [Tesla, Non-Proprietary] \\
		\hline Range/Max Charge Multiplier & [1, 1.25, 1.5] \\
		\hline Charger Reliability & [.75, .85, .95] \\
		\hline Risk Attitude & [Cautious, Neutral, Aggressive] \\
		\hline
	\end{tabular}
\end{table}

The parameters and levels selected reflect possible ways to mitigate the disparity between \glspl{bev} and \glspl{icev} and within both sets. The rationale behind the selection of levels is as follows. As mentioned previously, the Tesla and non-proprietary DC charging networks are, presently, effectively separate and unequal entities. The base vehicle models in this study are the Chevrolet Bolt and Tesla Model 3 both of which are low-end vehicles for their category with more expensive models coming with higher battery capacities and charge rates \cite{AFDC_EVs_2023}. The range of reliability rates was taken from references \cite{Rempel_2023} with an aspirational high reliability rate added. Finally, risk attitudes were modeled on the theoretical basis outlined in the Methods section using \eqref{eq:superquantile} with Cautious ($p_0 = .5,\ p_1 = 1$), Neutral ($p_0 = 0,\ p_1 = 1$), and Aggressive ($p_0 = 0,\ p_1 = .5$) drivers.

The metric used for comparison is long-trip accessibility as defined in \eqref{eq:a} where the cost used is total route time. In each case the optimal-path tree $P$ is computed using a stochastic, constrained implementation of Dijkstra's method as described in the Methods section with sample vectors of length 100. On average the difference between the accessibility score $A$ for the Model 3 and Bolt was 2.72 hours in favor of the Model 3 for the same combinations of experimental parameters. This difference is due to the superior range, charging rate, and DC charging infrastructure enjoyed by the Model 3. Linear regression was performed on the experimental parameters, interactions and results. Significant parameters ($\alpha = 0.05$) for the Bolt and Model 3 are listed in Tables \ref{tab:sp_bolt} and \ref{tab:sp_model_3} respectively.

\begin{table}[H]
	\centering
	\caption{Significant ($\alpha = 0.05$) terms from linear regression for Bolt.}
	\label{tab:sp_bolt}
	\begin{tabular}{|C{.6\linewidth}|C{.2\linewidth}|C{.2\linewidth}|}
		\hline Parameter & $\beta$ & p-value \\
		\hline {\small Intercept } & 8.557 & 0.000 \\
		\hline {\small Multiplier } & -2.127 & 0.000 \\
		\hline {\small Attitude[T.Neutral] } & 0.775 & 0.004 \\
		\hline {\small Attitude[T.Cautious] } & 1.188 & 0.000 \\
		\hline
	\end{tabular}
\end{table}

\begin{table}[H]
	\centering
	\caption{Significant ($\alpha = 0.05$) terms from linear regression for Model 3.}
	\label{tab:sp_model_3}
	\begin{tabular}{|C{.6\linewidth}|C{.2\linewidth}|C{.2\linewidth}|}
		\hline Parameter & $\beta$ & p-value \\
		\hline {\small Intercept } & 5.840 & 0.000 \\
		\hline {\small Multiplier } & -0.118 & 0.000 \\
		\hline
	\end{tabular}
\end{table}

The regression results support the intuitive conclusion that vehicle range is important in determining long-trip accessibility. A longer range vehicle will have more long-trip accessibility for a region of a given size. It is somewhat surprising that charger reliability was not significant for either vehicle/supply network combination. The lack of significance of reliability in the experimental range is due to the redundancy present in both supply networks. Where the Tesla DC charger network consists of stations with many chargers the non-proprietary network consists of smaller stations in closer proximity to one-another. Because there are more chargers total in the Tesla network and because these are redundant within stations rather than between stations the effect of non-functional chargers on user experience is lesser. A consequence of the more distributed non-proprietary network is that charger non-availability can catch a driver out and require a costly event relocation or even a tow. For this reason, driver risk attitude is more important for non-Tesla drivers for whom the trade-off between more direct routes with higher risk and less direct routes with lower risk is substantial. Overall the results indicate that the reliability of individual chargers is less important than their distribution and that the non-proprietary DC charging network is currently inferior to the Tesla DC charging network in California.

In the past, and still largely in the present, the Tesla DC charging network has been sequestered for the use of Tesla drivers. Recently, with the introduction of the NACS standard, this restriction has started to loosen. In the future, the Tesla and non-Tesla networks will function, increasingly, as one. A comparison of accessibility for each vehicle with each network independently, and the combined network is shown in Table \ref{tab:vehicles_networks}.

\begin{table}[H]
	\centering
	\caption{Accessibility for Bolt and Model 3 with neutral driver for various networks.}
	\label{tab:vehicles_networks}
	\begin{tabular}{|C{.4\linewidth}|C{.3\linewidth}|C{.3\linewidth}|}
		\hline Network & Bolt & Model 3 \\
		\hline Non-Proprietary & 9.49 & 6.32 \\
		\hline Tesla & 7.63 & 5.85 \\
		\hline Combined & 7.63 & 5.85 \\
		\hline
	\end{tabular}
\end{table}

The results show that the experience of Bolt drivers is improved by being able to use the Tesla DC charging network but does not redress the gap to Teslas using the Tesla network. Tesla drivers experiences are worsened by being only able to use the non-proprietary network but not to the same extent as those of Bolt drivers. It is notable that for the combined network both vehicles made exclusive use of Tesla DC charging stations. Given the demonstrated superiority of the Tesla network one might suppose that its opening will markedly improve the experience of non-Tesla drivers while failing to improve that of Tesla drivers. In fact, it may deteriorate Tesla driver experience due to higher demand at Tesla DC charging stations.

That the Tesla network is preferable to the non-proprietary network is due to redundancy. The Tesla network in California has roughly twice the number of chargers as the non-proprietary network with less than 30\% of the stations. Redundancy is not trivial to quantify for a network. Tesla chargers reinforce Tesla chargers within station but the distances between stations are large. Non-proprietary chargers reinforce between stations with smaller distances between. There is reason to suspect that even if the number of non-proprietary chargers were doubled but the distribution remained similar that Tesla's network would be preferable. The difference is informational. Where a driver finds a queue at a small station, that driver has uncertain information about the availability of chargers at proximate stations. Further, should that driver relocate to another station and find a similar or larger queue there is no mechanism to return to the vacated queue position at the previous station. This issue is more relevant for long-trips tahn it is for routine charging as the former generates less flexible charging demand. Policy makers should carefully consider which type of redundancy their policies encourage and what effects this will have. Quantifying the impacts of this informational issue is out of scope for this study and will be the subject of future research.

\section*{Conclusions}

Transportation accessibility is a very useful framework for policymakers and planners as it allows for the simultaneous consideration of land-use and transportation. For long trips, the land use component becomes exogenous in the short and medium term and the differentiating factor is transportation. Consideration of long road trips plays a role in vehicle selection out of proportion with their regularity encouraging continued \gls{icev} retention at the individual and household level. Quantifying the disadvantages of \glspl{bev} compared to \glspl{icev} is critical in addressing this issue. The framework and methodology developed are a valuable tool for th origination and evaluation of future policy.

In this study a framework for accessing long-trip accessibility for road vehicles in a region is developed. Using this framework and current data, a case study for the state of California was conducted comparing long-trip accessibility between \glspl{icev} and \glspl{bev} and between different \glspl{bev}. Results show that \glspl{icev} retain an advantage over \glspl{bev} of all types but this advantage is marginal for Tesla vehicles but major for low-end non-Tesla vehicles. the major difference is access to the Tesla DC charging network which is larger and differently structured compared to the non-proprietary networks. Specifically, the Tesla DC fast charging network is concentrated in large stations providing redundancy within station where the non-proprietary networks consist of distributed small stations providing redundancy between stations. The different approaches between Tesla and non-proprietary networks are due to the economic circumstances under which they developed. Policymakers and planners should note the differences between them and the effects these have on long-trip accessibility in considering new incentive programs.


 

\newpage

\printbibliography

\end{multicols}

\end{document}

%\subsection*{Random Graph Example}
%
%The optimal routing method in this study accounts for factors which influence road vehicle accessibility and derive from vehicular, infrastructural, and behavioral parameters. Important vehicular parameter for accessibility is range which determines which locations can be reached with and without charging. Important infrastructural parameters such as the number and locations of charging stations, usability rates, and expected delay times. Important behavioral parameters include risk attitude parameters as in \eqref{eq:superquantile}. These factors influence routing individually and via interactions as demonstrated on a randomly generated graph.
%
%A random graph of 100 nodes, 15 containing charging stations within a 100 km by 100 km square was generated with a random node selected as the origin. Edge probabilities were defined relative to a characteristic distance using
%
%\begin{equation}
%	P(e) = \exp(-L(e)/d)
%\end{equation}
%
%where $e$ is an edge, $L(e)$ is the distance of edge $e$, and $d$ is a characteristic distance set to 200 km for this example. In this example the vehicle has a range of 460 km and each charging station has one charger and a low vehicle arrival rate. Figure \ref{fig:interactions} show how $R$ is effected by different usability rates and risk attitudes.
%
%\end{multicols}
%
%\begin{figure}[H]
%	\centering
%	\begin{subfigure}{.4\linewidth}
	%		\centering\includegraphics[width = \linewidth]{figs/interactions_ha.png}
	%		\captionsetup{width=.8\linewidth}
	%		\caption{High charger usability and aggressive risk attitude}
	%	\end{subfigure}%
%	\begin{subfigure}{.4\linewidth}
	%		\centering\includegraphics[width = \linewidth]{figs/interactions_hc.png}
	%		\captionsetup{width=.8\linewidth}
	%		\caption{High charger usability and cautious risk attitude}
	%	\end{subfigure}
%	\begin{subfigure}{.4\linewidth}
	%		\centering\includegraphics[width = \linewidth]{figs/interactions_la.png}
	%		\captionsetup{width=.8\linewidth}
	%		\caption{Low charger usability and aggressive risk attitude}
	%	\end{subfigure}%
%	\begin{subfigure}{.4\linewidth}
	%		\centering\includegraphics[width = \linewidth]{figs/interactions_lc.png}
	%		\captionsetup{width=.8\linewidth}
	%		\caption{Low charger usability and cautious risk attitude}
	%	\end{subfigure}
%	\caption{Single-origin expected travel time trees with varying charger reliability and driver risk tolerance}
%	\label{fig:interactions}
%\end{figure}
%
%\begin{multicols}{2}
%
%As the example shows, it is possible for the impact of several factors to be additive when made disadvantageous simultaneously. Where the driver has an aggressive risk attitude, chargers are reliable, or both, accessibility is high. Where neither is the case accessibility is low. The differential can be dramatic and this highlights the importance of considering vehicular, infrastructural, and behavioral factors simultaneously to get a complete picture.