\documentclass[11pt]{article}
\usepackage[utf8]{inputenc}
\usepackage[
	letterpaper,
	left = .5in,
	right = .5in,
	top = 1in,
	bottom  = 1in
]{geometry}
\setlength{\columnsep}{.25in}
\usepackage{graphicx}
\usepackage{mathptmx}
\usepackage{float}
\usepackage[cmex10]{amsmath}
\usepackage{amsthm,amssymb}
\usepackage{url}
\urlstyle{same} 
\def\UrlBreaks{\do\/\do-}
\usepackage{breakurl}
\usepackage{fancybox}
\usepackage{breqn}
\usepackage{array}
\usepackage{caption}
\usepackage{subcaption}
%\usepackage{comment}
\usepackage[english]{babel}
\usepackage[acronym,nomain]{glossaries} % list of acronyms
%\usepackage{xurl}
\usepackage{multicol}
\usepackage{multirow}
\usepackage{mathptmx}
\usepackage{float}
\usepackage{lipsum}
%\usepackage{framed}
\usepackage[T1]{fontenc}
\usepackage[pdfpagelabels,pdfusetitle,colorlinks=false,pdfborder={0 0 0}]{hyperref}
%\usepackage{algorithm}
%\usepackage{algpseudocode}
%\usepackage{tabularx}
%\usepackage{wrapfig}
%\usepackage{enumitem}
\usepackage{paralist}

% draw a frame around given text
\newcommand{\framedtext}[1]{%
	\par%
	\noindent\fbox{%
		\parbox{\dimexpr\linewidth-2\fboxsep-2\fboxrule}{#1}%
	}%
}

\renewcommand{\arraystretch}{1.2}

\sloppy
\raggedbottom
\raggedcolumns

\newcolumntype{C}[1]{>{\centering\let\newline\\\arraybackslash\hspace{0pt}}m{#1-2\tabcolsep}}

\usepackage[%
backend=bibtex,     % biber or bibtex
%style=authoryear,    % Alphabeticalsch
style=numeric-comp,  % numerical-compressed
sorting=none,        % no sorting
sortcites=true,      % some other example options ...
block=none,
indexing=false,
citereset=none,
isbn=true,
url=true,
doi=true,           % prints doi
natbib=true,         % if you need natbib functions
]{biblatex}
\addbibresource{./sources/sources.bib,}  % better than \bibliography

\title{A Quantitative Framework for Assessing Long-Trip Transportation Accessibility for Road Vehicles}
\author{Aaron I. Rabinowitz, Vaishnavi Karanam, Radhika Gupta, Gil Tal\\(Alan Jenn?, Tom Bradley?, Susan Handy?, Yueyue Fan?)}
\date{}

\input{gloss}
\makeglossaries

\begin{document}

\maketitle

\section*{Abstract}

I'll write this later - could maybe use some help

%A well designed transportation system provides sufficient access from origins to destinations to accommodate household and business demand in an economy.  Increasingly, climate action goals require that more transportation load be shifted to less \gls{ghg} intensive modes among which are \glspl{bev}. While \glspl{bev} use the same roads as \glspl{icev} they draw energy from a separate network of stations which neither as robust as nor coincident to the \gls{icev} fueling network, a consequence of the different current and historical economics of both. Insufficiency and unreliability of public DC \gls{evse} which is primarily used for charging on long itineraries mean that \gls{bev} drivers, depending on vehicle range and risk attitude, may opt for less direct paths with lower charging risk, opt for a more \gls{ghg} intensive mode, or abandon an itinerary. Holistically, the transportation system provides less access to \glspl{bev} for distant pairs. There is also stark disparity between \gls{bev} types. This project develops methods and tools to optimize deployment of future \gls{evse} to mitigate the issue. Methods herein are based on range and charging risk sensitive optimal routing between O/D pairs subject to the locations and usability rates of \gls{evse}. These methods and tools may be used by policymakers to directly evaluate the impact of proposed policies on \gls{bev} transportation access.
\medskip

\begin{multicols}{2}

\section*{Introduction}

Developed economies require the continuous flow of persons and goods in astounding volumes. This dependency justifies investment in the development and maintenance of a transportation sector which is, in its own right, economically significant. It is intuitive that the function of all transportation systems is to enable the actualization of latent activity. It follows that transportation systems should be assessed on that basis and that well designed systems are those which efficiently enable latent flows. Encompassing this paradigm is the field of transportation accessibility. Transportation accessibility is, in short, the study of the related phenomena of how structural and individual factors create latent flows and how transportation systems accommodate them. More efficient built environments minimize the edge traversal costs which exist between demand and supply nodes. More efficient transportation allows for individuals and businesses to access more opportunities for the same cost. In the long term, co-optimization of land-use and transportation is vital to maximizing accessibility.

Transportation accessibility is, inherently, a network problem. Any method which seeks to quantify accessibility must define the nodes and edges which comprise the region in question. In the modern world, all nodes are, to a greater or lesser extent, connected. Scoping an accessibility analysis can be highly determinative of outcome. Researchers and planners have primarily utilized the concept of transportation accessibility as it applies to routine household behavior and local travel \cite{Handy_2020}. From the personal transportation perspective, access is defined as the ease with which individuals can reach the opportunities they desire subject to land-use, transportation infrastructure, temporal availability, and individual preference. Land-use dynamics determine the distribution and demand for amenities such as jobs and services at different locations. Transportation systems determine the edge traversal costs such as travel time, effort, and price which impede flows \cite{Geurs_2004}. Temporal availability of opportunities and transportation modes restrict the utility of opportunities and transportation modes. Finally, individual characteristics such as age, income, and education predict attraction to opportunities  and transportation modes \cite{Miller_2018} effecting their utility.

The US and similar western nations are unusually car-centric by global standards \cite{PrietoCuriel_2024} for personal transportation. The access provided by a road transportation system for \glspl{bev} is different than that for \glspl{icev} due to vehicular and supply network characteristics. Modern \glspl{bev} possess sufficient practical ranges to accomplish much daily of daily travel as shown in Figure \ref{fig:utility_factors} with data from \cite{NHTS_2017, NHTS_2022}. However, for long itineraries \glspl{icev} offer greater accessibility compared to \glspl{bev} due to vehicle ranges and the extensive availability of fueling stations in contrast to DC charging stations. Fueling stations are widely distributed across urban, suburban, and rural areas, ensuring that drivers have convenient access to refueling points wherever they travel. In contrast, the  charging network is less developed and distributed. This infrastructure gap poses challenges for \gls{bev} drivers, especially in remote or less densely populated areas, leading to concerns about range anxiety and limitations on travel options. Inadequate long-trip accessibility for \gls{bev} may result in cancellations or mode switches, often favoring \gls{icev} or air travel.

\begin{figure}[H]
	\centering
	\includegraphics[width = \linewidth]{figs/UF_2017_2022_km.png}
	\caption{Individual vehicle routine travel utility factors as a function of range by powertrain type for \gls{nhts} 2017 and 2022 editions.}
	\label{fig:utility_factors}
\end{figure}

Disparities between the fueling and DC charging networks result from differences in their economic models. Gas pumping equipment requires lower up-front costs than DC \gls{evse}, is cheaper to operate \cite{Gamage_2023}, and has been deployed for far longer. Gas is often sold at low margin with stations making most profit on convenience items. Nearly all light-duty \gls{icev} drivers source all of their fuel from public fueling stations regardless of travel behavior. \gls{bev} drivers are expected to, and currently do, source much of their electricity from AC supply equipment during long dwells, often at private chargers \cite{Hardman_2018}. Thus DC charging stations are subject to higher capital expenditure, lower revenue potential, and less accumulated investment. To overcome these disadvantages, public investments in DC charging infrastructure must be made judiciously. Evaluation methods for potential charging stations should consider their network-wide impact on accessibility considering vehicle types, equipment reliability, and driver risk attitudes.

This study introduces a novel methodology to assess transportation accessibility for long vehicular trips. The methodology measures accessibility by computing optimal-feasible travel routes for \gls{od} pairs using a stochastic routing algorithm subject to vehicle range limitations, supply infrastructure, and driver risk attitudes. This methodology is powertrain agnostic and can be used to directly compare accessibility for vehicles with different ranges and which rely on different supply networks. Additionally, a case study is presented for the state of California showing a comparison between \glspl{icev} and \glspl{bev} access for important \gls{od} pairs within the state. The methodology introduced, as well as the open-source code provided in the supplemental information is a valuable tool for planners and policymakers in originating and evaluating \gls{evse} deployment policies.

\section*{Transportation Accessibility}

Transportation accessibility has been studied as a tool for urban and regional planners since the middle of the 20\textsuperscript{th} century. Accessibility derives from the theory of population migration proposed by Ravenstein in 1885 \cite{Ravenstein_1885}. The movement of populations over a given time-scale can be analogized to Ohm's Law. In this analogy push and pull factors determine the "voltage" separation, traversal difficulty is the "resistance" and the resulting "current" is the flow. The field of transportation accessibility uses this framework to study regional efficiency considering both voltage and resistance simultaneously. The accepted definition of transportation accessibility is the ease with which individuals can access relevant opportunities subject to the transportation system in the relevant area. Thus, accessibility is a framework which encompasses voltage factors such as land use and temporal availability, resistance factors such as transportation system design, and universal factors such as personal preference \cite{Geurs_2004}. Literature provides four essential frameworks for computing access as surveyed in \cite{Handy_1997, Kwan_1998, Geurs_2004, Miller_2018, Handy_2020} and discussed below.

Much of the difference between methodologies is in the selection of opportunities. Individuals are assumed to need or desire location-specific opportunities such as employment and physical retail. However, there may be several near-equivalents for any given opportunity type. The simplest methods for selecting opportunities are based on nearest proximity \cite{Wachs_1973, Vickerman_1974}. Proximity methods consider that a person has a level of access to a given need as determined by that person's proximity to the closest relevant opportunity. These methods do not account for heterogeneity within an opportunity category nor for the benefits of redundancy within an opportunity category. The inverse are isocost methods wherein a person is said to have access as determined by the number of opportunities available within a given isocost polygon. This method has the drawback of not considering the differences in edge traversal cost for \gls{od} pairs within the isocost region. These methods have been used widely \cite{Easa_1993} due to their computational lightness and form the basis for modern big-data methods such as the US DOE's Mobility Energy Productivity metric \cite{Hou_2019}.

Proximity and isocost methods are easy to compute because they treat redundancy arbitrarily. In practice, equivalent and near-equivalent opportunities compete with one-another if sufficiently proximate or if the paths required to reach them overlap \cite{Stouffer_1940}. Gravity/entropy methods \cite{Noulas_2012, Jung_2008} address this shortcoming. These methods are so called as they concern the cumulative effect of opportunities for a given origin on the basis of demand over proximity (gravity) or information content (entropy). Such methods were first formalized into a quantitative framework in 1959 \cite{Hansen_1959} as a generalization of previous methodology for quantifying the efficiency of urban land use. Gravity/entropy methods define accessibility as the intensity of the possibility for interaction. Implicit in the formulation of gravity/entropy methods is that every opportunity has some effect on every individual, even if negligible, and the effect of any one opportunity is determined by its network position.

Proximity and gravity/entropy methods rely on the assumption that traversal cost is the primary factor determining individuals decision to select one opportunity from among a set of similar entities. While this is certainly true if the difference in traversal cost is large enough it is not, altogether, obvious what the threshold of disambiguation is for a given individual. Thus, researchers have proposed to use Discrete Choice Modeling \cite{Ben_Akiva_1985} to explain revealed choices wherein ease-of-access is one of several possible factors in determining the utility of a given opportunity for an individual \cite{Cevero_1995, Shen_1998, Karst_2003}.

There are, thus, a variety of methods which can be used to quantify the accessibility of a given region with varying computational and data requirements. The relationship between land-use, transportation, and demography is circular rather than linear. Which method one chooses for an analysis reflects the scope and purpose of that analysis. Definition of scope can be difficult and can lead to self-defeating policies \cite{Handy_1996}. This study is concerned with the effects of electrification on long-trip accessibility for road vehicle users. This scope simplifies opportunity selection. It is necessary that a transportation system provide for access between large population centers within a region of interest and to those in adjacent regions.  This study is focused on non-routine regional travel rather than routine local travel as this is where supply infrastructure becomes important. It should be noted that the method is valid for all travel scales. Methodology is developed in the following section.

\section*{Methods}

The focus of this study is long-trip accessibility for road vehicles. The metric proposed reflects the ease with which a given individual, driving a given vehicle, can access the important locations in a region from one-another. It is assumed that travelers, when considering non-routine long trips have a specific destination in mind and would not consider nearly equivalent and/or equidistant locations as fungible. Rather, having determined to travel to a given location the traveler will then select a mode. Accessibility is a function of demand and impedance. Demand can be estimated by considering the relative attractiveness of given cities as a function of population, economy, or some other metric. Impedance can be quantified by calculating the mode and person specific lowest-cost-paths for all pairs of selected locations within the region and taking the average. Example costs which might be used are travel time and travel price.

A transportation accessibility metric should have land-use, transportation, temporal, and individual components \cite{Karst_2003}. The metric proposed contains all required components. The land-use within a region has two principle effects. First, for a multi-city region, peripheral cities should experience worse regional access than central cities. Second, geographically large and/or sparse regions should experience worse overall accessibility than compact regions. The transportation infrastructure determines the efficacy of various modes. Where only vehicular travel is concerned, the mode choice is reduced to vehicle choice. Vehicle specific infrastructure is not necessarily the same for all vehicles of a given fuel type but is necessarily different between fuel types creating effectively separate modes. The temporal component arises from the schedules of public transportation services and road traffic patterns. Finally the individual component is the traveler risk attitude and route cost weights.

\subsection*{Metric Definition}

For region $R$ with a set of $N$ selected locations $O = \{O_1, O_2, \dots, O_N\}$ and a corresponding set of importance weights $W = \{W_0, W_1, \dots, W_N\}$, the accessibility of the region is quantified by the metric $A_{R}$ as

\begin{equation}
	A_{R} = \frac{1}{N^2}\sum_{i = 0}^{N} \sum_{j = 0 }^{N} \frac{CW_iW_j}{Z_{ij}\underline{W}} \label{eq:regional_accessibility}
\end{equation}

\noindent where $Z_{ij}$ is the cost of the optimal path from $O_i$ to $O_j$, $\underline{W} = \sum_{i = 0}^{N} W_i$, and C is a constant. The location specific regional accessibility $A_{R,i}$ can be computed for region $R$ and origin $i$ as

\begin{equation}
	A_{R,i} = \frac{1}{N}\sum_{j = 0 }^{N} \frac{CW_iW_j}{Z_{ij}\underline{W}} \label{eq:specific_regional_accessibility}
\end{equation}

for any origin $O_i \in O$. $A_{R}$ is a weighted mean of flows resulting from the potential interactions between important locations in a region and the difficulty to traverse the arcs between them. Generally, $W$ will be the set of population masses for each location $M = \{M_0, M_1, \dots, M_N\}$. A physical meaning for $A_R$ may be attained if the units of $W$ are [persons] and the units of $Z$ are [hours]. In this case the units of $A_R$ would be [persons-hours\textsuperscript{-1}] reflecting the average long-trip flow in the region. It is expected that $C$ will be dimensionless and fractional. In essence, a given location might be presumed to attract a small portion of the populations of other cities on any given day in proportion to its population, economy, or other factors. However, the attraction of remaining in one's current location will be quite strong, limiting potential travel. A strongly connected region will have higher average accessibility than a weakly integrated one. All else being equal, regions which are more populated/productive, more compact, and enjoy better transportation networks will experience higher average flow. Regional Accessibility refers the the all-pairs average as in \eqref{eq:regional_accessibility} and Specific Regional Accessibility refers to the single-origin average as in \eqref{eq:specific_regional_accessibility}.

This study is primarily concerned with the transportation system which contributes directly to arc traversal cost rather than arc demand. Thus a second metric, Regional Impedance is defined as

\begin{equation}
	Z_{R} = \frac{1}{N^2}\sum_{i = 0}^{N} \sum_{j = 0 }^{N} W_iW_jZ_{ij} \label{eq:regional_impedance}
\end{equation}

\noindent where $W = \{W_0, W_1, \dots, W_N\}$ is the set of importance weights assigned to each origin and destination. Specific Regional Impedance is defined as

\begin{equation}
	Z_{R,i} = \frac{1}{N}\sum_{j = 0 }^{N} W_iW_jZ_{ij} \label{eq:specific_regional_impedance}
\end{equation}

for a single origin. The Regional Impedance is the weighted average cost of traversing an arc between important locations in a region. Impedance is strictly a measure of travel costs imposed by the transportation system and is usage rate independent. For the purposes of computing impedance it is beneficial to utilize relative weights ($W_i = NM_i / \sum_{j = 0}^{N} M_j\ \forall M_i \in M$). For a region $R$ with population masses $M$ and arc traversal costs $Z$, if each value in $M$ was doubled and nothing else changed $A_R$ would increase and $Z_R$ would remain the same. Any increase to arc costs $Z$ would decrease $A_R$ and increase $Z_R$.

\subsection*{Metric Computation}

Powered vehicles are range-limited due to the finite capacity of their \glspl{ess}. In order to traverse an \gls{od} arc whose energy requirement is greater than a vehicle's \gls{ess} capacity, the vehicle must stop at a supply station. In practical terms, the limit is defined by the vehicle's \gls{ess} capacity, starting \gls{soc}, desired finishing \gls{soc}, and the driver's low \gls{soc} tolerance. Because supply events add time to a trip, they will usually be avoided where possible. For sufficiently long trips, where at least one supply event is necessary, computing the shortest-time path requires considering the time added during supply events and the time required to deviate to supply stations. For \glspl{icev}, supply events are brief and supply stations are ubiquitous in most areas of the developed world. Thus, conventional routing services often neglect supply considerations. For \glspl{bev}, supply events are relatively lengthy and DC supply stations are comparatively rare. Ignoring supply events when computing a shortest-time path for a \gls{bev} carries non-negligible risk of a lengthy stop or an out-of-charge event. For this reason, dedicated \gls{bev} routing services such as A Better Route Planner compute routes considering supply events.

All drivers deal with uncertainty and latency issues when computing an optimal route. Certain categories of disruption such as traffic, accidents, supply equipment functionality, and supply equipment local demand cannot be precisely known at the start of the trip. This uncertainty will cause different drivers to evaluate the same arc differently and may result in the selection of different routes. Compounding uncertainty is latency. Although a driver may have access to on the current state of roads and stations to be encountered in the future this may provide little information about conditions upon arrival. Without the ability to reserve a slot at a supply station, drivers cannot be certain of equipment availability until they physically arrive at the station. As such, drivers have to, and do actually, optimize an expectation of route costs when selecting routes. Such optimization may take the form of electing to take a longer ring highway to avoid city traffic, stopping for fuel more often in a sparsely populated area, sitting through a traffic jam rather than diverting, or any number of other strategies. This mental process is modeled via stochastic optimization.

\subsubsection*{Stochastic Optimal Routing}

The purpose of stochastic optimal routing is to find lowest-expected-cost paths from origin $i \in V$ to a set of destinations $D \in V$ on graph $G = \{V, E\}$. The output is tree $P$ containing the optimal-feasible paths from the origin to the selected destinations. The objective of routing on arc $(i,j)\ i, j \in V$ is

\begin{equation}
	\min_{U \in \overline({U}_{i,j})}\ \mathbb{E}[J(S_0, U)]
\end{equation}

where

\begin{equation}
	J(U) = \sum_{k = 0}^M \Phi_k(S_0, U)
\end{equation}

s.t.

\begin{gather}	
	b^k_l \leq \mathbb{E}\left[\int_0^t \Phi_k(S_0, U)dt\right] \leq b^k_u\\
	\mathbb{E}\left[\int_0^T \Phi_k(S_0, U)dt\right] \geq b^k_f\\
	 t \in [0, T]\quad k = 1, 2, \dots, M
\end{gather}

\noindent where $T$ is the final value of time for a route, $S$ is the state vector, $S_0$ is the initial values of the states, $U$ is a control (route) between $i$ and $j$, $\overline{U}_{i,j}$ is the set of possible routes between $i$, and $j$, $\Phi$ is the set of cost functions, $b^k_l$ and $b^k_u$ are the upper and lower bounds for state $k$ respectively, and $b^k_f$ is the final state minimum value for state $k$. $\mathbb{E}$ denotes an expectation. State vector $S$ is initialized and stored as vectors containing $N$ discreet variables. A distribution $D$ for a the state vector at any node and time-step can be computed from a histogram of the values. States can be changed at nodes and edges. Routes are considered feasible if state expectations remain within set bounds. Comparison between routes is performed using cost expectation. Routing depends on vehicular, infrastructural, and individual models.


\subsubsection*{Vehicle Model}

Vehicles effect routing due to their range limits and supply methods. The vehicle model used herein is highly simplified due to the inexact nature of the problem. Vehicles are modeled as storing energy and consuming energy at a constant rate per unit distance driven. More exact information on road conditions, traffic conditions, and atmospheric conditions among others can be used to compute edge-specific efficiencies. Vehicular parameters are listed in Table \ref{tab:param_veh}.

\begin{table}[H]
	\centering
	\caption{Vehicle Parameters for Routing}
	\label{tab:param_veh}
	\begin{tabular}{|C{\linewidth*3/8}|C{\linewidth*3/8}|C{\linewidth*2/8}|}
		\hline Parameter & Description & Unit \\
		\hline \gls{ess} Capacity & Accessible energy storage capacity & [kWh] \\
		\hline Energy Consumption & Energy required to move the vehicle & [kJ/km] \\
		\hline Supply Rate & \gls{ess} maximum energy addition rate & [kW] \\
		\hline Supply Limits & \gls{soc} bounds for supply & [-] \\
		\hline
	\end{tabular}
\end{table}

\subsubsection*{Supply Infrastructure Network Model}

Supply networks effect routing both in structure and in the characteristics of individual stations. In this paper, the supply network refers to all stations which the vehicle can utilize rather than the industry definition which covers stations operated by a single \gls{cpo}. Networks ultimately consist of individual supply ports (chargers or fuel pumps) and serve geographically and temporally distributed demand. A network consisting of more than one port can develop redundancy either by concentrating ports in a single confined space "in-station" or via a more evenly distributed approach "between-station". Network redundancy also varies by location with "thinner" and "thicker" coverage areas.

Vehicle optimal routing serves to find the lowest-cost-path between two points on a road atlas. A road atlas is a graph with a large number of nodes and a low edge-to-node ratio. When considering a multi-leg itinerary, such as when solving the Traveling Salesman Problem or Vehicle Routing Problem, the atlas need not be stored in entirety. Only the locations of and relationships between relevant nodes are necessary. The graph formed from the relevant nodes and their relationships is a reduced subgraph with a low number of nodes and relatively high edges-per-node and cycles-per-node ratios. A reduced subgraph is defined as follows. For a graph $G = \{V, E\}$ where $V$ is the set of nodes and $E$ is the set of edges, a reduced subgraph $\hat{G} = \{\hat{V}, \hat{E}\}$ can be computed where $\hat{V} \subseteq V$ and $\hat{E} = \{P_{ij}\ \forall\ (i, j)\ \in\ \hat{E}\}$ is the set of shortest-paths between all nodes in $\hat{V}$. An example of a reduced subgraph is shown in Figure \ref{fig:reduced_subgraph}.

\begin{figure}[H]
	\centering
	\begin{subfigure}[t]{.5\linewidth}
		\centering\captionsetup{width = .8\linewidth}
		\includegraphics[width = \linewidth]{figs/full_graph.png}
	\end{subfigure}%
	\begin{subfigure}[t]{.5\linewidth}
		\centering\captionsetup{width = .8\linewidth}
		\includegraphics[width = \linewidth]{figs/reduced_graph.png}
	\end{subfigure}
	\caption{Example original graph (left) containing locations and an atlas and reduced subgraph (right) containing locations and arcs.}
	\label{fig:reduced_subgraph}
\end{figure}

For a given vehicle type, the \gls{sng} is the reduced subgraph containing the trip origin, trip destination, and all supply stations reasonably likely to be utilized. In-station redundancy for a \gls{sng} is a nodal parameter. In-station  redundancy $\xi_{is}$ for any node $v \in V$ on \gls{sng} $G = \{V, E\}$ is the number of ports at node $v$. Between-station redundancy $\xi_{bs}$ for an \gls{sng} is a clique parameter. Between-station redundancy for any clique $C \subseteq V$ on \gls{sng} $G = \{V, E\}$ is the sum of the ports at nodes $v \in C$. A more meaningful result for $\xi_{bs}$ may be returned by limiting the cost of edges considered. For example, $\xi'_{bs}$ may be computed for $G' = \{V, E'\}$ where $E' \subseteq E$ contains all edges of less than 10 minutes drive time.

The \gls{sng} is the graph on which long vehicle trips should be optimized if supply events are expensive and/or constraining. The relevant \gls{sng} for \glspl{icev} and \gls{bev} are neither equivalent nor isomorphic. Different vehicles within a given powertrain type may also have different \glspl{sng} but this is far more common for \glspl{ev} than \glspl{icev}. The \gls{sng} informs routing by providing a set of possible paths between origins and destinations. The structure of a network may make for a greater or lesser number of available paths depending on the number and location of stations.

\subsubsection*{Supply Station Network Model}

Supply station characteristics significantly impact the expected cost of utilizing a given station. Supply stations are defined by number of ports, reliability of ports, maximum supply rate of ports, durations of supply events, and arrivals ratio. In combination, these factors determine the likelihood of a port being usable and available as well as the likely duration of queue if no port is usable and available. Supply station parameters are listed in Table \ref{tab:param_supply}.

\begin{table}[H]
	\centering
	\caption{Supply Station Parameters for Routing}
	\label{tab:param_supply}
	\begin{tabular}{|C{\linewidth*2/8}|C{\linewidth*3/8}|C{\linewidth*3/8}|}
		\hline Parameter & Description & Unit \\
		\hline Arrivals Ratio & Arrival rate at station per port and unit of time & [cars/port/hour] \\
		\hline Supply Rate & Maximum rate of energy supply & [kW] \\
		\hline Supply Energy & Expected energy supplied to vehicles & [kWh] \\
		\hline Ports & Number of chargers/pumps at a station which can be used simultaneously & [-] \\
		\hline Reliability & Percentage of the time that a given pump will be usable & [-] \\ 
		\hline
	\end{tabular}
\end{table}

Information on ports is taken from \gls{afdc} \cite{afdc_2023}, and information on chargers is taken from \cite{Rempel_2023}. Arrivals ratio reflects the scale of vehicle arrivals at the station compared to the ability of the station to process vehicles. Arrivals ratio is defined as

\begin{equation}
	\rho = \frac{\lambda}{\mu s}\label{eq:rho}
\end{equation}

where $\lambda$ is the number of vehicle arrivals in a given time period, $\mu$ is the expected number of vehicles that can be served by a port in the same time period, and $s$ is the number of ports. As an example, if a station of 5 ports supplies a mean of 36 kWh per vehicle at a mean rate of 80 kW with 3 minutes between events and receives 10 vehicles per hour then $\rho = 1$ for the station. Unfortunately, with the exception of the number of ports and the maximum supply rate, little of this information is available before arrival. As a result, arrivals ratio, supply energy, and reliability must be estimated. To reflect this uncertainty, in this study distributions are used for arrivals ratio and supply energy. The delays which result from DC charging station congestion are modeled as M/M/s queues and are demonstrated in \ref{fig:expected_delay}.

\end{multicols}

%\begin{figure}[H]
%	\centering
%	\includegraphics[width = \linewidth]{figs/expected_delay.png}
%	\caption{Expected queuing time at charging stations. Low arrival rate is 10 - 60 minutes between arrivals. High arrival rate is 1 - 10 minutes between arrivals. All charge rates are 80 kW and charge events are 60 $\pm$ 15 kWh.}
%	\label{fig:expected_delay}
%\end{figure}


\begin{figure}[H]
	\centering
	
	\begin{subfigure}{.33\linewidth}
		\centering
		\includegraphics[width = \linewidth]{figs/expected_delay_contourf_1.png}
		\caption{Mean arrival ratio of $\rho = 1$}
	\end{subfigure}%
	\begin{subfigure}{.33\linewidth}
		\centering
		\includegraphics[width = \linewidth]{figs/expected_delay_contourf_2.png}
		\caption{Mean arrival ratio of $\rho = 2$}
	\end{subfigure}%
	\begin{subfigure}{.33\linewidth}
		\centering
		\includegraphics[width = \linewidth]{figs/expected_delay_contourf_3.png}
		\caption{Mean arrival ratio of $\rho = 3$}
	\end{subfigure}
	
	\caption{Effects of supply station saturation on expected queue times by in-station redundancy. Contours of PDFs of expected queue times are computed for stations with between 1 and 20 chargers with different arrival ratios. Arrival ratio is defined in equation \eqref{eq:rho}. Higher arrival ratios saturate stations but stations with greater redundancy can better handle high demand per charger.}
	\label{fig:expected_delay}
	
\end{figure}

\begin{multicols}{2}

As can be seen in Figure \ref{fig:expected_delay}, arrivals ratio and in-station redundancy each have a major impact on expected delay time. When $\rho \leq 1$ a decently sized station should be able to accommodate demand with minimal delay. As arrivals increase congestion becomes unavoidable and even large stations can become saturated. Unreliability of equipment can, effectively, reduce in-station redundancy making queues more likely. Without better information, a driver will have to make an educated guess as to how busy a station is likely to be and how many of its listed chargers will be functional. Getting either factor wrong could have serious consequences if the driver does not have the option to easily reach a different station. Even if another station is reachable the driver is unlikely to have better information for that station.

\subsubsection*{Drivers}

Given the same underlying circumstances, different drivers will evaluate route costs differently. In a basic sense, drivers will weight several factors such as time, money, distance, and complexity differently. Where any important factor is not known precisely drivers will consider a range of outcomes and decide based on an expectation. Driver risk attitude concerns what range of outcomes will be used to compute expected cost. Risk attitude is modeled using a superquantile risk function defined as

\begin{equation}
	S(D, p_0, p_1) = \frac{1}{p_1 - p_0}\int_{p_0}^{p_1}Q(D, \alpha)\ d\alpha \label{eq:superquantile}
\end{equation}

\noindent where $D$ is a distribution, $p_0$ and $p_1$ are the boundaries of the range of probabilities considered in the expectation, and $Q$ is the quantile function of $D$. The superquantile is, thus, the mean value of a distributed quantity within a range of probability. $S(D, 0, 1)$ reduces to the mean of $D$. Drivers with an aggressive risk attitude will consider a low range of probabilities. Drivers with a neutral attitude will consider a central range of probabilities. Drivers with a cautious attitude will consider a high range of probabilities. Driver parameters are listed in Table \ref{tab:param_driver}.

\begin{table}[H]
	\centering
	\caption{Supply Station Parameters for Routing}
	\label{tab:param_driver}
	\begin{tabular}{|C{\linewidth*3/8}|C{\linewidth*3/8}|C{\linewidth*2/8}|}
		\hline Parameter & Description & Unit \\
		\hline Route Cost Weights $W$ & Set of multipliers for route costs to be used in computation of weighted sum & [-] \\
		\hline Probability Range $(p_0, p_1)$ & Range of probabilities for superquantile function & [-] \\
		\hline
	\end{tabular}
\end{table}

Where a quantity is to be maximized the Probability Range parameters can be adjusted as $p_0' = 1 - p_1$ and $p_1' = 1 - p_0$. This is particularly applicable when evaluating states such as \gls{soc} where a cautious driver will underestimate the quantity and an aggressive driver will overestimate the quantity. The driver model serves to bias the routing by selecting a subset of information to use in optimization. As such it reflects individual perception. The results of the routing can be interpreted through the same bias, a different bias, or no bias. In this study, evaluation will be conducted on the basis of un-biased interpretation of costs.

\subsection*{Randomly Generated Example}

The methods and metrics defined above are demonstrated using a vehicle on a randomly generated \gls{sng}. In this example, a \gls{sng} containing 15 locations and 85 stations distributed across a 1000 km square is generated with randomized node locations. Each location has a population of one million persons. Each station contains between 1 and 5 ports. Edges exist between all nodes and are assigned Pythagorean distances. All edges are assumed to be traversed at 105 kmh. The supply stations deliver an average of 45 kWh per vehicle at an average rate of 80 kW. The arrivals ratio is sampled from $N(1, 0.25)$ The vehicle used has a relatively low range of 262 km. Two drivers are modeled. A cautious driver is modeled using ($p_0 = 0.9$, $p_1 = 1.0$) and a aggressive driver is modeled using ($p_0 = 0$, $p_1 = 0.1$). Additionally port reliability at the stations is modeled as 75\% in a low reliability scenario and 95\% in a high reliability scenario.

\begin{figure}[H]
	\centering
	\begin{subfigure}[t]{.5\linewidth}
		\centering\captionsetup{width = .8\linewidth}
		\includegraphics[width = \linewidth]{figs/random_example_high_reliability_aggressive_perceived.png}
		\caption{High reliability, aggressive}
	\end{subfigure}%
	\begin{subfigure}[t]{.5\linewidth}
		\centering\captionsetup{width = .8\linewidth}
		\includegraphics[width = \linewidth]{figs/random_example_high_reliability_cautious_perceived.png}
		\caption{High reliability, cautious}
	\end{subfigure}
	\begin{subfigure}[t]{.5\linewidth}
		\centering\captionsetup{width = .8\linewidth}
		\includegraphics[width = \linewidth]{figs/random_example_low_reliability_aggressive_perceived.png}
		\caption{Low reliability, aggressive}
	\end{subfigure}%
	\begin{subfigure}[t]{.5\linewidth}
		\centering\captionsetup{width = .8\linewidth}
		\includegraphics[width = \linewidth]{figs/random_example_low_reliability_cautious_perceived.png}
		\caption{Low reliability, cautious}
	\end{subfigure}
	\caption{Perceived Specific Regional Impedance by risk attitude and port reliability.}
	\label{fig:perceived_srta_random_perceived}
\end{figure}

The scenarios presented in Figure \ref{fig:perceived_srta_random_perceived} consider Specific Regional Impedance as perceived by the driver. The aggressive driver is only concerned with the best 10\% of outcomes where the cautious driver is only concerned with the worst 10\% of outcomes. The differences in perceived costs-to-travel are quite stark between the aggressive and cautious driver in both cases but this difference is larger when reliability is low. Drivers may operate off of perception but a regional transportation authority may be more concerned with mean outcomes. The neutral expectations ($p_0 = 0$, $p_1 = 1$) of the routes taken by the drivers are shown in Figure \ref{fig:perceived_srta_random_actual}.

\begin{figure}[H]
	\centering
	\begin{subfigure}[t]{.5\linewidth}
		\centering\captionsetup{width = .8\linewidth}
		\includegraphics[width = \linewidth]{figs/random_example_high_reliability_aggressive_actual.png}
		\caption{High reliability, aggressive}
	\end{subfigure}%
	\begin{subfigure}[t]{.5\linewidth}
		\centering\captionsetup{width = .8\linewidth}
		\includegraphics[width = \linewidth]{figs/random_example_high_reliability_cautious_actual.png}
		\caption{High reliability, cautious}
	\end{subfigure}
	\begin{subfigure}[t]{.5\linewidth}
		\centering\captionsetup{width = .8\linewidth}
		\includegraphics[width = \linewidth]{figs/random_example_low_reliability_aggressive_actual.png}
		\caption{Low reliability, aggressive}
	\end{subfigure}%
	\begin{subfigure}[t]{.5\linewidth}
		\centering\captionsetup{width = .8\linewidth}
		\includegraphics[width = \linewidth]{figs/random_example_low_reliability_cautious_actual.png}
		\caption{Low reliability, cautious}
	\end{subfigure}
	\caption{Neutral expectations of Specific Regional Impedance by risk attitude and port reliability.}
	\label{fig:perceived_srta_random_actual}
\end{figure}

The neutral expectations tell a different story. For the different levels of reliability, neutral expectations of impedance are similar for the routes selected by the aggressive and cautious drivers. The differences between biased and neutral perception are easy to understand. Both drivers selected routes based on a subset of the information and these routes are non-optimal when all information is considered. Drivers may adjust their priors over time but policy can only control the fundamentals. Thus, it is recommended to consider bias in route planning but neutral expectation in evaluation.

The composition of the routes taken by the drivers in the randomly generated example demonstrate an interesting dynamic as seen in Table \ref{tab:distances_redundancy}. As in-station redundancy is highly important in determining expected queuing times cautious drivers opt, increasingly for stations with higher in-station redundancy at the cost of taking less direct routes.

\begin{table}[H]
	\centering
	\caption{Average route distances and chargers per station utilized for example scenarios}
	\label{tab:distances_redundancy}
	\begin{tabular}{|C{\linewidth / 3}|C{\linewidth / 3}|C{\linewidth / 3}|}
		\hline Reliability & Risk-Attitude & Chargers per Station Utilized \\
		\hline High & Aggressive & 3.774 \\
		\hline High & Cautious & 4.258 \\
		\hline Low & Aggressive & 4.032 \\
		\hline Low & Cautious & 4.471 \\
		\hline
	\end{tabular}
\end{table}

\section*{California Case Study}

The randomly generated example is informative but does not reflect any actual \gls{sng}. In order to see effects on a more representative basis a case study is performed on the state of California using information on the states DC EV \gls{sng} with modes of common \glspl{bev} which enjoy different levels of access.

\subsection*{Background}

The state of California is geographically large and contains major population centers distributed across the state. Major road transportation corridors form connections between the state's population centers and with population centers in adjacent states. This case study concerns the long-trip accessibility of California's road transportation network. For the purposes of analysis, 15 important cities in California and adjacent states were selected. Non-California locations are represented by the most-likely departure locations at the California state line. Because the state-line locations are midpoints the required arrival \gls{soc} at these points is set at 50\% where it is set at 20\% for all others. These locations are enumerated in Table \ref{tab:locations}.

\begin{table}[H]
	\centering
	\caption{Locations Considered for Long Trip Accessibility}
	\label{tab:locations}
	\begin{tabular}{|C{.2\linewidth}|C{.5\linewidth}|C{.3\linewidth}|}
		\hline Index & Location & Approximate Metropolitan Population \\
		\hline 0 & Crescent City & 6,000 \\
		\hline 1 & Yreka & 8,000 \\
		\hline 2 & Redding & 180,000 \\
		\hline 3 & Chico & 212,000 \\
		\hline 4 & Reno (State Line) & 491,000 \\
		\hline 5 & Sacramento & 2,397,000 \\
		\hline 6 & Stockton & 779,000 \\
		\hline 7 & San Francisco & 4,620,000 \\
		\hline 8 & San Jose & 2,001,000 \\
		\hline 9 & Fresno & 1,009,000 \\
		\hline 10 & Las Vegas (State Line) & 2,953,000 \\
		\hline 11 & Bakersfield & 909,000 \\
		\hline 12 & Los Angeles & 12,488,000 \\
		\hline 13 & Phoenix (State Line) & 4,948,000 \\
		\hline 14 & San Diego & 3,276,000 \\
		\hline
	\end{tabular}
\end{table}

For long trips, \glspl{bev} will rely on DC charging stations. The locations of all DC charging stations in California are available from AFDC \cite{afdc_2023}. In may 2024 \gls{afdc} listed 2,149 active stations with at least 1 DC charger. This number is somewhat misleading as certain networks report each charger as an individual station even if within line-of-sight of one-another. After merging all stations of the same network which are within 100 meters direct distance of each other the number of stations becomes 1,689. California's DC charging stations include proprietary (vehicle \gls{oem} owned and operated) stations such as Tesla Superchargers and the Rivian Adventure network as well as non-proprietary stations such as those operated by ChargePoint, Electrify America, eVgo, and others. The selected locations and DC charging stations are mapped in Figure \ref{fig:california_stations}.

\end{multicols}

\begin{figure}[H]
	\begin{subfigure}{\linewidth/3}
		\centering
		\includegraphics[width = \linewidth]{figs/California_SNG_T.png}
		\caption{Tesla Stations}
	\end{subfigure}%
	\begin{subfigure}{\linewidth/3}
		\centering
		\includegraphics[width = \linewidth]{figs/California_SNG_NT.png}
		\caption{Non-Tesla Stations}
	\end{subfigure}
	\begin{subfigure}{\linewidth/3}
		\centering
		\includegraphics[width = \linewidth]{figs/California_SNG_Corridor.png}
		\caption{Corridor Stations}
	\end{subfigure}
	\caption{California DC charging stations from \gls{afdc} (May 2024)}
	\label{fig:california_stations}
\end{figure}

\begin{multicols}{2}

The proprietary and non-proprietary networks in California are neither equivalent nor isomorphic. There are 419 Tesla and 16 Rivian DC charging stations in the state as compared to 1,254 non-proprietary DC charging stations. In practice, many of these stations will be of little use for long distance travel being located far away from primary and secondary roads. Considering only those stations within 1 km of a highway as "corridor" chargers, there are a total of 500 corridor DC charging stations. Of the corridor stations, 156 are Tesla stations, 7 are Rivian stations, and 344 are non-Proprietary stations.

The non-Tesla networks overwhelmingly use combination CCS/ChaDeMo chargers as defined by SAE J1772 \cite{SAE_J1772} which reflect the ports on the overwhelming number of non-Tesla \glspl{bev}. By contrast, Tesla chargers and vehicles use the NACS standard as defined by SAE J3400 \cite{SAE_J3400}. The Tesla and non-Tesla systems are increasingly interoperable with the aid of adapters but should be considered separately in the present. Tesla drivers use Tesla DC chargers almost exclusively \cite{Visaria_2022}. The Rivian Adventure network is technically interoperable with other J1772 vehicles but is set aside for the exclusive use of Rivian vehicles. The purpose of the Rivian Adventure network serves to allow for Rivian vehicles to charge in remote locations and is not intended to be relied upon exclusively.

The difference between the Tesla DC charging network and the non-Tesla networks extends from function to form. Built out as an investment to entice sales of Tesla vehicles and, until recently, exclusive to them, the Tesla network is technically superior with higher maximum charging rates and more reliable chargers \cite{Rempel_2023, Kozumplik_2022}. The Tesla network is mainly composed of high redundancy stations. Non-proprietary networks have, so far, been utilization and subsidy driven \cite{Gamage_2023} and are widely distributed with low redundancy stations. A stark contrast is seen when examining the ratios of chargers to stations. In California there are 403 Tesla DC charging stations with a total of 7,101 DC chargers for an average of 16.9 chargers per station. Among non-proprietary networks there are a total of 1,254 stations with 4,129 chargers for an average of 3.3 per station. Redundancies for Tesla and non-proprietary networks are shown in Figure \ref{fig:network_histograms}. 


\begin{figure}[H]
	\centering
	\includegraphics[width = \linewidth]{figs/California_RIS_Hist.png}
	\caption{Survival functions for in-station redundancy for Tesla and other DC charger networks in California}
	\label{fig:network_histograms}
\end{figure}

The Tesla DC charging network develops redundancy primarily in-station where the non-proprietary networks develop redundancy primarily between stations. Non-Tesla chargers are also more likely to be sighted in urban areas suggesting a desire to capture local as well as corridor travel demand. Tesla stations are more often sighted along travel corridors suggesting a focus on enabling long distance travel. In more remote parts of California the proprietary networks nearly match the non-proprietary networks in station numbers. In-station redundancies for DC charging networks in California can be found in Figures \ref{fig:ris_top_networks} and \ref{fig:ris_top_networks_corridor} in the Appendix. Summary statistics for DC charging networks in California can be found in Tables \ref{tab:summary_statistics_afdc} and \ref{tab:summary_statistics_afdc_corridor} in the appendix.

California \glspl{icev} utilize a third and completely separate network of supply stations. There are estimated to be over 8,000 gasoline stations in California \cite{CEC_2022} and these are widely and proportionally distributed. Because no public database for the locations of gasoline stations in the state exists, and due to their ubiquity it is assumed in this study that \gls{icev} driver optimal paths will not be effected by fueling station availability. For this reason, \glspl{icev} are, herein, assumed to take the "direct" path between cities where \glspl{bev} need to find optimal paths on their \glspl{sng}.

\subsection*{Experiment}

In order to understand the effects of vehicular, infrastructural, and behavioral parameters on road-trip accessibility an experiment was carried out on randomly generated combinations. As a baseline, three \glspl{icev} were also modeled. These \glspl{icev} represent different levels of efficiency present in the \gls{icev} fleet. \gls{ess} capacity numbers are pulled from manufacturer websites and energy consumption rates are computed from EPA highway fuel economy ratings \cite{DOE_EPA_2024}. Although substantially less efficient than equivalent \glspl{bev} the comparatively high specific energy of liquid petroleum allows for \glspl{icev} to have higher maximum ranges. \gls{icev} supply infrastructure is modeled to dispense fuel at the normal US rate of 7 gallons per minute which is an equivalent energy supply rate of 14.15 MW. When refueling, the Prius. Golf, and Pacifica, add highway range at rates of 631, 462, and 282 km per minute respectively. The \gls{icev} models and corresponding Regional Impedance values are shown in Table \ref{tab:icev_models}.

\begin{table}[H]
	\centering
	\caption{\gls{icev} models}
	\label{tab:icev_models}
	\begin{tabular}{|C{.45\linewidth}|C{.3\linewidth}|C{.25\linewidth}|}
		\hline Vehicle Model & Parameter & Value \\
		\cline{1-3} & \gls{ess} Capacity & 381 [kWh] \\
		\cline{2-3} & Energy Consumption & 1,346 [kJ/km] \\
		\cline{2-3} 2024 Toyota Prius & Full-Tank Range & 1,018 [km] \\
		\cline{2-3} & California Regional Impedance & 4.131 [hours] \\
		\cline{1-3} & \gls{ess} Capacity & 445 [kWh] \\
		\cline{2-3} & Energy Consumption & 1,839 [kJ/km] \\
		\cline{2-3} 2024 Volkswagen Golf & Full-Tank Range & 871 [km] \\
		\cline{2-3} & California Regional Impedance & 4.151 [hours] \\
		\cline{1-3} & \gls{ess} Capacity & 640 [kWh] \\
		\cline{2-3} & Energy Consumption & 3,015 [kJ/kM] \\
		\cline{2-3} 2024 Chrysler Pacifica & Full-Tank Range & 764 [km] \\
		\cline{2-3} & California Regional Impedance & 4.161 [hours] \\
		\hline
	\end{tabular}
\end{table}

Regional Impedance for \glspl{icev} was computed under the assumption of petroleum supply infrastructure ubiquity. As such, \glspl{icev} were given the "direct" path between locations with stop times added where additional range was needed. For each necessary stop, time was added for refueling to full as well as 10 minutes to divert from the road and handle the transaction prior to refueling. Additionally, drivers of the \glspl{icev} were assumed to keep a 10\% buffer of remaining range. The \glspl{icev} each had similar road-trip accessibility scores of roughly 5.5 hours. The longest arc considered in Crescent City (Location 0) to Phoenix - State Line (Location 13) which is roughly 1,530 km just exceeding double the usable range of the Pacifica. The Prius, Golf, and Pacifica required averages of 0.43, 0.31, and 0.19 supply stops per route respectively.

500 random scenarios were generated by uniform random sampling of the parameters listed in Table \ref{tab:experimental_parameters} and run on three \glspl{sng} as described in \ref{tab:experimental_sngs}. All randomly sampled \glspl{bev} in this study are assumed to have an energy consumption rate of 608 kJ/km this being the EPA energy consumption rate of a Tesla Model 3 in highway operation \cite{DOE_EPA_2024}. Highway operation is assumed herein due to the focus on long trips. Thus \gls{bev} full-charge ranges will be between 237 and 711 km. Vehicles are assumed to fast charge only up to 80\% \gls{soc} in order to remain in the constant current range. When charging, sampled \glspl{bev} add highway range at a rate between 4.9 and 19.7 km per minute. Risk attitude was modeled as in \eqref{eq:superquantile} with the range centered around the mean parameter $\overline{p}$ where $p_0 = \overline{p} - .1$ and $p_1 = \overline{p} + .1$. 

\begin{table}[H]
	\centering
	\caption{Parameters and ranges for experiment.}
	\label{tab:experimental_parameters}
	\begin{tabular}{|C{\linewidth/2}|C{\linewidth/2}|}
		\hline Parameter & Range \\
		\hline \gls{ess} Capacity & [40 kWh, 120 kWh] \\
		\hline \gls{ess} Max Charge Rate & [50 kW, 200 kW] \\
		\hline Driver Risk-Attitude Mean & [.1, .9] \\
		\hline \gls{evse} Reliability & [.5, 1] \\
		\hline Station Arrival Ratio Mean & [1, 3] \\
		\hline
	\end{tabular}
\end{table}

\begin{table}[H]
	\centering
	\caption{\glspl{sng} used in experiment.}
	\label{tab:experimental_sngs}
	\begin{tabular}{|C{\linewidth/3}|C{\linewidth*2/3}|}
		\hline Label & Networks Included \\
		\hline Combined & All stations \\
		\hline Tesla & Only Tesla stations \\
		\hline Non-Tesla & All non-Tesla stations \\
		\hline
	\end{tabular}
\end{table}

Outputs were processed to compute a population weighted Regional Impedance as in \eqref{eq:regional_impedance}. Impedance was selected as the metric of evaluation rather than accessibility as travel demand for each arc was the same for all cases. Linear regression was performed on the results of the random experiment. Using as output, the neutral expectation of road-trip accessibility for each of the 500 randomly sampled vehicles on each of the \glspl{sng}. Significant parameters from the regression are shown in Figure \ref{fig:significant_parameters}.

\begin{figure}[H]
	\centering
	\includegraphics[width = \linewidth]{figs/significant_parameters.png}
	\caption{Coefficients for significant parameters from linear regression}
	\label{fig:significant_parameters}
\end{figure}

Regression details are provided in Tables \ref{tab:regression_anova} and \ref{tab:regression_coefficients} in the Appendix. The regression analysis shows that vehicular, infrastructural, and behavioral parameters have significant impacts on road-trip accessibility. The vehicular parameters of capacity and charge rate have the predictable effect of reducing expected travel times. Capacity being the more more important parameter is explicable as higher capacity vehicles offer the ability to stop less frequently. Saving an entire charging event can be very impactful in a region the size of California. Among the infrastructure parameters, higher reliability contributed to lower travel times where higher arrival ratios contributed to higher travel times. Both reliability and arrival ratio primarily effect queuing times as, even with 50\% equipment reliability, most stations have sufficient redundancy to guarantee at least one operational charger. The range of arrivals ratios considered goes from normal to swamped and long queues can be expected for high arrival ratios even at high redundancy stations. Risk attitude was also significant in determining outcomes as those drivers with more cautious risk attitudes will tend to maintain higher \gls{soc} and utilize more reliable and redundant routes. finally, access to the entire network is better than restriction to just part of it but being restricted to only Tesla stations should be slightly less damaging than access to only non-Tesla stations. Boxplots of expected travel times are shown in Figure \ref{fig:networks_boxplots}.

\begin{figure}[H]
	\centering
	\includegraphics[width = \linewidth]{figs/Networks_Boxplots_Weighted_Impedance.png}
	\caption{Boxplots of experiment outputs by \gls{sng}}
	\label{fig:networks_boxplots}
\end{figure}

While the differences between the \glspl{bev} on the basis of \gls{sng} access are observed to be slight, the differences between the means of each and the worst of the \glspl{icev} is quite large. The difference is roughly 45 minutes much of which can be explained by the shorter ranges and longer supply times inherent to the \glspl{bev}. In the best of circumstances, \glspl{bev} can reach near parity with \glspl{icev}. However, \glspl{bev} impedance will, nearly always, be substantially higher than \gls{icev} impedance due to the necessity of charging events. Some will argue that this difference is unimportant as \gls{bev} drivers can charge their vehicles while stopping for meals or during other natural breaks. This logic has been used to show that \glspl{bev} may approach convenience parity with \glspl{icev} in reasonable circumstances \cite{Dixon_2020}. However, where such long breaks are optional for \gls{icev} drivers, they are mandatory for \gls{bev} drivers and must be taken at specific points throughout the trip to coincide with charging. This loss of optionality must be accounted an inconvenience even for drivers who accustomed to long breaks.

Most of the difference in impedance is due to charging and queuing times while only a small part is due to route alterations. Isolating the driving times by subtracting supply and setup times, the differences between \glspl{bev} and \glspl{icev} \glspl{sng} are substantially smaller than those in total trip time as shown in Figure \ref{fig:networks_boxplots_driving}. It should, however, be noted that there is a long tail on the driving time distributions due to the occasional need to take a significantly altered route to confidently reach a remote destination.

\begin{figure}[H]
	\centering
	\includegraphics[width = \linewidth]{figs/Networks_Boxplots_Weighted_Impedance_Driving.png}
	\caption{Boxplots of experiment outputs by \gls{sng} (driving time only)}
	\label{fig:networks_boxplots_driving}
\end{figure}

Different parts of the state will have different experiences. Residents of large population centers which are proximate to other large population centers will expect to have lower travel needs than those in more remote areas. In general, the impedance disparities associated with different vehicular, infrastructural, and behavioral characteristics will also scale with impedance. As California is a large state with a diversity of land use and road infrastructure volume, there is a pronounced difference in Specific Regional Impedance as shown in Figure \ref{fig:networks_boxplots_locations}.

\end{multicols}

\begin{figure}[H]
	\centering
	\includegraphics[width = \linewidth]{figs/Networks_Boxplots_Weighted_Specific_Impedance_2.png}
	\caption{Boxplots of experiment outputs on combined \gls{sng} for each origin in California}
	\label{fig:networks_boxplots_locations}
\end{figure}

\begin{multicols}{2}
	
\section*{Discussion}

The optimal routes generated in this study only selected corridor chargers. Some insight into the utility provided by each network can be gained by looking into utilization rates for networks and stations when using the combined \gls{sng}. Ratio of stations utilized at least once to total corridor stations in the combined \gls{sng} for each network is shown in Figure \ref{fig:utilization_rates}. Log of utilization (number of times utilized in random experiment) for corridor stations in the combined \gls{sng} is displayed in Figure \ref{fig:utilized_stations}.

\begin{figure}[H]
	\centering
	\includegraphics[width = \linewidth]{figs/corridor_station_utilization.png}
	\caption{Corridor DC charging network utilization rates}
	\label{fig:utilization_rates}
\end{figure}

\begin{figure}[H]
	\centering
	\includegraphics[width = \linewidth]{figs/California_SNG_Utilization.png}
	\caption{Corridor DC charging station log utilization}
	\label{fig:utilized_stations}
\end{figure}

It follows intuition that those chargers most useful for drivers are those located between cities on busy transportation corridors and this is backed up in this study. It also follows intuition that those chargers most likely to see use are those with higher redundancy in-station and lower redundancy in-corridor. In this sense Tesla stations are well positioned to absorb traffic should they become fully open to the general light duty \gls{ev} fleet. When one single entity has control over the locations of all stations in a network, that entity can concentrate chargers to maximize in-station redundancy and minimize between-station redundancy. That entity can, then strategically locate the highly concentrated stations. The result should be a network which experiences high utilization at each station and can benefit from economies of scale. When each network is developed by an independent actor these actors must build out their networks under the fear that a competitor could build a competing station near any of their stations at any time. The risk of building a large station only to lose business to a nearby competitor combined with the opportunity cost of not using limited capital to do the same inevitably results in a more inefficient combined network. However, many of the issues inherent with the distributed structure could be mitigated if the uncertainty and latency issues inherent to such a network could be mitigated. These issues can, possibly, be mitigated by an integrated status reporting and reservation system.


Often, the case for \glspl{bev} is made on an economic basis as \gls{bev} may have lower levelized costs of driving. This study focuses on travel times and routing did not optimize for cost. Partly, this is because at present there is no publicly available data on energy costs at a granular station-level for either gasoline or DC fast charging stations. Nevertheless, some sense of the relative economics of long-trip travel can be attained by examining energy costs per km. Energy costs vary substantially by region in the US with California being the most expensive. Energy costs around the time of writing are shown in Table \ref{tab:energy_costs}.

\begin{table}[H]
	\centering
	\caption{Residential electricity and petroleum average prices USD}
	\label{tab:energy_costs}
	\begin{tabular}{|C{.31\linewidth}|C{.23\linewidth}|C{.23\linewidth}|C{.23\linewidth}|}
		\hline Source & US & California & Percentage Increase \\
		\hline Petroleum [gallon] & 3.609 & 5.138 & 42.37 \\
		\hline Residential Electricity [kWh] & 0.1668 & 0.3247 & 94.66 \\
		\hline Transportation Electricity [kWh] & 0.1520 & 0.1191 & 27.62 \\
		\hline DC Fast Charging (Estimated) [kWh] & 0.35 - 0.50 & 0.35 - 0.60 & 0 - 20 \\
		\hline
	\end{tabular}
\end{table}

Petroleum prices are from AAA \cite{AAA_2024} and electricity prices are from EIA \cite{EIA_2024}. DC fast charging pricing schemes display much heterogeneity and may not be as easily accounted as metered electricity prices. An Ad-Hoc Text Mining study performed on over 90,000 recorded PlugShare events from 2019 and 2021 found the mode of DC fast charging prices to be in the range of 0.3 and 0.4 USD per kWh \cite{Trinko_2021}. Prices did not significantly correlate with local energy prices. In the same time period California residential electricity increased from 0.1995 USD per kWh to 0.2282 USD per kWh and transportation electricity increased from 0.0891 to 0.1179 USD per kWh. By comparison with 2024 electricity prices, one would expect prices in the range of 0.35 and 0.60 USD per kWh for DC fast charging in California and 0.35 to 0.5 in the US, ranges backed by informal reporting \cite{CalTrans_2024, Sowder_2024}. Thus, expected energy costs per highway km traveled can be computed and are shown in Table \ref{tab:expected_energy_costs_per_km}.

\begin{table}[H]
	\centering
	\caption{Expected energy costs per highway km traveled in US cents.}
	\label{tab:expected_energy_costs_per_km}
	\begin{tabular}{|C{\linewidth / 4}|C{\linewidth / 4}|C{\linewidth / 4}|C{\linewidth / 4}|}
		\hline Vehicle & Source & US Price & CA Price \\
		\hline Prius & Petroleum & 4.00 & 5.70 \\
		\hline Golf & Petroleum & 5.47 & 7.78 \\
		\hline Pacifica & Petroleum & 8.97 & 12.77 \\
		\hline \gls{bev} & Residential Electricity & 2.82 & 5.48 \\
		\hline \gls{bev} & DC Fast Charging & 5.91 - 8.44 & 5.91 - 10.13 \\
		\hline
	\end{tabular}
\end{table}

In the US, DC fast charging a \gls{bev} presents no appreciable economic benefit over fueling an efficient \gls{icev}. In much of the US, home-charging a \gls{bev} provides cost savings for daily travel and the initial part of a long trip. This is not the case in California where residential electricity is, on average, nearly twice as expensive as in the US as a whole.

If \glspl{bev} are a worse option than efficient \glspl{icev} for long trips on a time basis and no better on an energy cost basis this will make them less appealing to customers who value the ability to make long road trips. That customers seem to so highly value these uncommon events is a continuing source of frustration for \gls{bev} advocates. Negative perceptions of \gls{bev} long trip utility on consumer stated preference were found to be quite important in the late 2010s \cite{Skippon_2016, Hardman_2016, Franke_2017, Schmalfuss_2017}. In the intervening time period \gls{bev} ranges and maximum charge rates have markedly increased. Nevertheless, negative perceptions related to long trip utility persist for purchasers \cite{Bhat_2022, Paradies_2023, Corradi_2023, Philip_2023} and \gls{bev} range is a significant factor in determining usage share of \glspl{bev} in multi-vehicle household fleets \cite{Chakraborty_2022}. In the same time period, a massive build-out of DC charging infrastructure has taken place yet is not evident that the increased presence of DC charging infrastructure changes perceptions \cite{Hoogland_2023}.

As shown in Figure \ref{fig:utility_factors}, a \gls{bev} with 300 km of range can accomplish 80\% of US daily itineraries on a single charge. Fundamentally, the reality is that home or work charging leads to operational costs that often cheaper and rarely more expensive than petroleum. Similarly, home and work charging can lead to convenience benefits for \glspl{bev} as compared to \glspl{icev} \cite{Rabinowitz_2023} as \glspl{icev} require trips and trip deviations to reach supply stations. In theory, the trade-off of lower cost routine travel for higher cost long-distance travel is one which works in favor of \glspl{bev}. This is the logic which underpins the "charging pyramid" model which places long dwell charging events at its base and corridor DC fast charging events at the top. This model is also, to some degree, self-reinforcing. Because \gls{bev} drivers prefer AC charging, DC charging infrastructure has limited revenue potential leading to lower network capacity. Lower network capacity, in turn, leads to the perception that the network is inadequate and should be avoided.

The charging pyramid implies a different way of thinking about the role of car travel as a subset an individual's travel needs. Personal travel is inherently multi-modal and, for many \gls{od} arcs and many individuals, the cost differential between different modes is within the threshold of disambiguation. The strengths and weaknesses of \glspl{bev} as compared to \glspl{icev} may shift more short trips away from local transit and towards cars while shifting more long trips away from cars to air travel and inter-city transit. \glspl{bev} will only be one part of future mobility, unable to meet transportation needs or environmental goals on their own. Investments in \gls{bev} corridor infrastructure should be considered alongside investments into other low carbon inter-city transit modes.

\section*{Conclusions}

I'll write this later - could maybe use some help

 

\newpage

\printbibliography

\appendix

\begin{figure}[H]
	\centering
	\includegraphics[width = \linewidth]{figs/California_RIS_SF_All.png}
	\caption{In-station redundancy for DC Charging networks in California}
	\label{fig:ris_top_networks}
\end{figure}

\begin{figure}[H]
	\centering
	\includegraphics[width = \linewidth]{figs/California_RIS_SF_Corridor.png}
	\caption{In-station redundancy for DC Charging networks in California}
	\label{fig:ris_top_networks_corridor}
\end{figure}


%\begin{figure}[H]
%	\centering
%	\includegraphics[width = \linewidth]{figs/California_RBS_300_SF_All.png}
%	\caption{In-station redundancy for DC Charging networks in California}
%	\label{fig:rbs_300_top_8_networks}
%\end{figure}
%
%
%\begin{figure}[H]
%	\centering
%	\includegraphics[width = \linewidth]{figs/California_RBS_600_SF_All.png}
%	\caption{In-station redundancy for DC Charging networks in California}
%	\label{fig:rbs_600_top_8_networks}
%\end{figure}

\begin{table}[H]
	\centering
	\caption{Summary statistics for California DC charging networks from \gls{afdc}}
	\label{tab:summary_statistics_afdc}
	\begin{tabular}{|C{.46\linewidth}|C{.18\linewidth}|C{.18\linewidth}|C{.18\linewidth}|}
		\hline Network & Chargers & Stations & Chargers per Station \\
		\hline Non-Networked & 288 & 51 & 5.6 \\
		\hline Tesla & 2753 & 156 & 17.6 \\
		\hline Electrify America & 526 & 77 & 6.8 \\
		\hline EV Connect & 59 & 19 & 3.1 \\
		\hline ChargePoint Network & 186 & 79 & 2.4 \\
		\hline Volta & 2 & 2 & 1.0 \\
		\hline EVCS & 41 & 11 & 3.7 \\
		\hline SHELL_RECHARGE & 39 & 11 & 3.5 \\
		\hline EVGATEWAY & 21 & 5 & 4.2 \\
		\hline eVgo Network & 332 & 63 & 5.3 \\
		\hline BP_PULSE & 3 & 2 & 1.5 \\
		\hline POWERFLEX & 12 & 3 & 4.0 \\
		\hline FLO & 1 & 1 & 1.0 \\
		\hline EVRANGE & 11 & 3 & 3.7 \\
		\hline RIVIAN_ADVENTURE & 14 & 7 & 2.0 \\
		\hline CIRCLE_K & 16 & 3 & 5.3 \\
		\hline CHARGENET & 7 & 1 & 7.0 \\
		\hline Blink Network & 2 & 2 & 1.0 \\
		\hline NOODOE & 2 & 1 & 2.0 \\
		\hline LOOP & 6 & 2 & 3.0 \\
		\hline 7CHARGE & 4 & 1 & 4.0 \\
		\hline
	\end{tabular}
\end{table}

\begin{table}[H]
	\centering
	\caption{Summary statistics for California DC charging networks from \gls{afdc} (corridor stations)}
	\label{tab:summary_statistics_afdc_corridor}
	\begin{tabular}{|C{.46\linewidth}|C{.18\linewidth}|C{.18\linewidth}|C{.18\linewidth}|}
		\hline Network & Chargers & Stations & Chargers per Station \\
		\hline Non-Networked & 274 & 50 & 5.5 \\
		\hline Tesla & 2745 & 155 & 17.7 \\
		\hline Electrify America & 518 & 76 & 6.8 \\
		\hline EV Connect & 58 & 18 & 3.2 \\
		\hline ChargePoint Network & 184 & 78 & 2.4 \\
		\hline Volta & 1 & 1 & 1.0 \\
		\hline EVCS & 35 & 10 & 3.5 \\
		\hline SHELL_RECHARGE & 37 & 10 & 3.7 \\
		\hline EVGATEWAY & 19 & 4 & 4.8 \\
		\hline eVgo Network & 329 & 62 & 5.3 \\
		\hline BP_PULSE & 1 & 1 & 1.0 \\
		\hline POWERFLEX & 10 & 2 & 5.0 \\
		\hline EVRANGE & 5 & 2 & 2.5 \\
		\hline RIVIAN_ADVENTURE & 12 & 6 & 2.0 \\
		\hline CIRCLE_K & 12 & 2 & 6.0 \\
		\hline Blink Network & 1 & 1 & 1.0 \\
		\hline LOOP & 2 & 1 & 2.0 \\
		\hline
	\end{tabular}
\end{table}

\begin{table}[H]
	\centering
	\caption{Linear Regression Analysis ANOVA}
	\label{tab:regression_anova}
	\begin{tabular}{|C{.25\linewidth}|C{.25\linewidth}|C{.25\linewidth}|C{.25\linewidth}|}
		\hline R & R-Squared & Adjusted R-Squared & Std. Error \\
		\hline 0.930 & 0.865 & 0.860 & 0.000 \\
		\hline
		\hline Category & Sum of Squares & DOF & Mean Squares \\
		\hline Model & 405.979 & 47 & 8.638 \\
		\hline Error & 63.621 & 1452 & 0.044 \\
		\hline Total & 469.601 & 1499 & 0.313 \\
		\hline  \multicolumn{2}{|c|}{$F$} &  \multicolumn{2}{c|}{$P(>F)$}  \\
		\hline  \multicolumn{2}{|c|}{197.138} &  \multicolumn{2}{c|}{0.000}  \\
		\hline
	\end{tabular}
\end{table}

\begin{table}[H]
	\centering
	\caption{Linear Regression Analysis Coefficients and P-Values}
	\label{tab:regression_coefficients}
	\begin{tabular}{|C{.5\linewidth}|C{.25\linewidth}|C{.25\linewidth}|}
		\hline Parameter & Coefficient & P-Value \\
		\hline {\small Intercept } & 5.245 & 0.000 \\
		\hline {\small capacity } & -1.417 & 0.000 \\
		\hline {\small Tesla Only } & 0.037 & 0.005 \\
		\hline {\small charge_rate } & -0.273 & 0.000 \\
		\hline {\small reliability } & -0.504 & 0.000 \\
		\hline {\small arrival_ratio } & 0.701 & 0.000 \\
		\hline {\small risk_attitude } & 0.669 & 0.000 \\
		\hline {\small Non-Tesla Only } & 0.102 & 0.000 \\
		\hline
	\end{tabular}
\end{table}

%\begin{figure}[H]
%	\centering
%	\includegraphics[width = \linewidth]{figs/expected_delay_contourf_1.png}
%	\caption{PDFs of expected delay time for station with mean arrival ratio of 1, constant mean charge energy delivered of 45 kWh, and different in-station redundancy.}
%	\label{fig:expected_delays_coutnours_1}
%\end{figure}
%
%\begin{figure}[H]
%	\centering
%	\includegraphics[width = \linewidth]{figs/expected_delay_contourf_2.png}
%	\caption{PDFs of expected delay time for station with mean arrival ratio of 2, constant mean charge energy delivered of 45 kWh, and different in-station redundancy.}
%	\label{fig:expected_delays_coutnours_2}
%\end{figure}
%
%\begin{figure}[H]
%	\centering
%	\includegraphics[width = \linewidth]{figs/expected_delay_contourf_3.png}
%	\caption{PDFs of expected delay time for station with mean arrival ratio of 3, constant mean charge energy delivered of 45 kWh, and different in-station redundancy.}
%	\label{fig:expected_delays_coutnours_3}
%\end{figure}

\end{multicols}

\end{document}