\documentclass[11pt]{article}
\usepackage[utf8]{inputenc}
\usepackage[
	letterpaper,
	left = .5in,
	right = .5in,
	top = 1in,
	bottom  = 1in
]{geometry}
\setlength{\columnsep}{.25in}
\usepackage{graphicx}
\usepackage{mathptmx}
\usepackage{float}
\usepackage[cmex10]{amsmath}
\usepackage{amsthm,amssymb}
\usepackage{url}
\urlstyle{same} 
\def\UrlBreaks{\do\/\do-}
\usepackage{breakurl}
\usepackage{fancybox}
\usepackage{breqn}
\usepackage{array}
\usepackage{caption}
\usepackage{subcaption}
\usepackage{comment}
\usepackage[english]{babel}
\usepackage[acronym,nomain]{glossaries} % list of acronyms
\usepackage{xurl}
\usepackage{multicol}
\usepackage{multirow}
\usepackage{mathptmx}
\usepackage{float}
\usepackage{lipsum}
\usepackage{framed}
\usepackage[T1]{fontenc}
\usepackage[pdfpagelabels,pdfusetitle,colorlinks=false,pdfborder={0 0 0}]{hyperref}
\usepackage{algorithm}
\usepackage{algpseudocode}
\usepackage{tabularx}
\usepackage{wrapfig}
%\usepackage{enumitem}
\usepackage{paralist}

% draw a frame around given text
\newcommand{\framedtext}[1]{%
	\par%
	\noindent\fbox{%
		\parbox{\dimexpr\linewidth-2\fboxsep-2\fboxrule}{#1}%
	}%
}

\renewcommand{\arraystretch}{1.2}

\sloppy
\raggedbottom

\newcolumntype{C}[1]{>{\centering\let\newline\\\arraybackslash\hspace{0pt}}m{#1-2\tabcolsep}}

\usepackage[%
backend=bibtex,     % biber or bibtex
%style=authoryear,    % Alphabeticalsch
style=numeric-comp,  % numerical-compressed
sorting=none,        % no sorting
sortcites=true,      % some other example options ...
block=none,
indexing=false,
citereset=none,
isbn=true,
url=true,
doi=true,           % prints doi
natbib=true,         % if you need natbib functions
]{biblatex}
\addbibresource{./sources/sources.bib,}  % better than \bibliography

\title{A Quantitative Framework for Assessing Long-Trip Transportation Accessibility for Road Vehicles}
\author{Aaron I. Rabinowitz, Vaishnavi Karanam, Radhika Gupta, Gil Tal\\(Alan Jenn?, Tom Bradley?, Susan Handy?, Yueyue Fan?)}
\date{}

\input{gloss}
\makeglossaries

\begin{document}

\maketitle

\section*{Abstract}

A well designed transportation system provides sufficient access from origins to destinations to accommodate household and business demand in an economy.  Increasingly, climate action goals require that more transportation load be shifted to less \gls{ghg} intensive modes among which are \glspl{bev}. While \glspl{bev} use the same roads as \glspl{icev} they draw energy from a separate network of stations which neither as robust as nor coincident to the \gls{icev} fueling network, a consequence of the different current and historical economics of both. Insufficiency and unreliability of public DC \gls{evse} which is primarily used for charging on long itineraries mean that \gls{bev} drivers, depending on vehicle range and risk attitude, may opt for less direct paths with lower charging risk, opt for a more \gls{ghg} intensive mode, or abandon an itinerary. Holistically, the transportation system provides less access to \glspl{bev} for distant pairs. There is also stark disparity between \gls{bev} types. This project develops methods and tools to optimize deployment of future \gls{evse} to mitigate the issue. Methods herein are based on range and charging risk sensitive optimal routing between O/D pairs subject to the locations and usability rates of \gls{evse}. These methods and tools may be used by policymakers to directly evaluate the impact of proposed policies on \gls{bev} transportation access.
\medskip

\begin{multicols}{2}

\section*{Introduction}

Developed economies require the continuous flow of persons and goods in astounding volumes. This dependency justifies investment in the development and maintenance of a transportation sector which is, in its own right, economically significant. It is intuitive that the function of all transportation systems is to enable the actualization of latent activity. It follows that transportation systems should be assessed on that basis and that well designed systems are those which efficiently enable latent flows. Encompassing this paradigm is the field of transportation accessibility. Transportation accessibility is, in short, the study of the related phenomena of how structural and individual factors create latent flows and how transportation systems accommodate them. More efficient built environments minimize the edge traversal costs which exist between demand and supply nodes. More efficient transportation allows for individuals and businesses to access more opportunities for the same cost. In the long term, co-optimization of land-use and transportation is vital to maximizing accessibility.

Transportation accessibility is, inherently, a network problem. Any method which seeks to quantify accessibility must define the nodes and edges which comprise the region in question. In the modern world, all nodes are, to a greater or lesser extent, connected. Scoping an accessibility analysis can be highly determinative of outcome. Researchers and planners have primarily utilized the concept of transportation accessibility as it applies to routine household behavior and local travel \cite{Handy_2020}. From the personal transportation perspective, access is defined as the ease with which individuals can reach the opportunities they desire subject to land-use, transportation infrastructure, temporal availability, and individual preference. Land-use dynamics determine the distribution and demand for amenities such as jobs and services at different locations. Transportation systems determine the edge traversal costs such as travel time, effort, and price which impede flows \cite{Geurs_2004}. Temporal availability of opportunities and transportation modes restrict the utility of opportunities and transportation modes. Finally, individual characteristics such as age, income, and education predict attraction to opportunities  and transportation modes \cite{Miller_2018} effecting their utility.

The US and similar western nations are unusually car-centric by global standards \cite{PrietoCuriel_2024} for personal transportation. The access provided by a road transportation system for \glspl{bev} is different than that for \glspl{icev} due to vehicular and supply network characteristics. Modern \glspl{bev} possess sufficient practical ranges to accomplish much daily of daily travel as shown in Figure \ref{fig:utility_factors} with data from \cite{NHTS_2017, NHTS_2022}. However, for long itineraries \glspl{icev} offer greater accessibility compared to \glspl{bev} due to vehicle ranges and the extensive availability of fueling stations in contrast to DC charging stations. Fueling stations are widely distributed across urban, suburban, and rural areas, ensuring that drivers have convenient access to refueling points wherever they travel. In contrast, the  charging network is less developed and distributed. This infrastructure gap poses challenges for \gls{bev} drivers, especially in remote or less densely populated areas, leading to concerns about range anxiety and limitations on travel options. Inadequate long-trip accessibility for \gls{bev} may result in cancellations or mode switches, often favoring \gls{icev} or air travel.

\begin{figure}[H]
	\centering
	\includegraphics[width = \linewidth]{figs/UF_2017_2022_km.png}
	\caption{Individual vehicle routine travel utility factors as a function of range by powertrain type for \gls{nhts} 2017 and 2022 editions.}
	\label{fig:utility_factors}
\end{figure}

Disparities between the fueling and DC charging networks result from differences in their economic models. Gas pumping equipment requires lower up-front costs than DC \gls{evse}, is cheaper to operate \cite{Gamage_2023}, and has been deployed for far longer. Gas is often sold at low margin with stations making most profit on convenience items. Nearly all light-duty \gls{icev} drivers source all of their fuel from public fueling stations regardless of travel behavior. \gls{bev} drivers are expected to, and currently do, source much of their electricity from AC supply equipment during long dwells, often at private chargers \cite{Hardman_2018}. Thus DC charging stations are subject to higher capital expenditure, lower revenue potential, and less accumulated investment. To overcome these disadvantages, public investments in DC charging infrastructure must be made judiciously. Evaluation methods for potential charging stations should consider their network-wide impact on accessibility considering vehicle types, equipment reliability, and driver risk attitudes.

This study introduces a novel methodology to assess transportation accessibility for long vehicular trips. The methodology measures accessibility by computing optimal-feasible travel routes for \gls{od} pairs using a stochastic routing algorithm subject to vehicle range limitations, supply infrastructure, and driver risk attitudes. This methodology is powertrain agnostic and can be used to directly compare accessibility for vehicles with different ranges and which rely on different supply networks. Additionally, a case study is presented for the state of California showing a comparison between \glspl{icev} and \glspl{bev} access for important \gls{od} pairs within the state. The methodology introduced, as well as the open-source code provided in the supplemental information is a valuable tool for planners and policymakers in originating and evaluating \gls{evse} deployment policies.

\section*{Transportation Accessibility}

Transportation accessibility has been studied as a tool for urban and regional planners since the middle of the 20\textsuperscript{th} century. Accessibility derives from the theory of population migration proposed by Ravenstein in 1885 \cite{Ravenstein_1885}. The movement of populations over a given time-scale can be analogized to DC power flow. In this analogy push and pull factors determine the "voltage" separation, traversal difficulty is the "resistance" and the resulting "current" is the flow. The field of transportation accessibility uses this framework to study regional efficiency considering both voltage and resistance simultaneously. The accepted definition of transportation accessibility is the ease with which individuals can access relevant opportunities subject to the transportation system in the relevant area. Thus, accessibility is a framework which encompasses voltage factors such as land use and temporal availability, resistance factors such as transportation system design, and universal factors such as personal preference \cite{Geurs_2004}. Literature provides four essential frameworks for computing access as surveyed in \cite{Handy_1997, Kwan_1998, Geurs_2004, Miller_2018, Handy_2020} and discussed below.

Much of the difference between methodologies is in the selection of opportunities. Individuals are assumed to need or desire location-specific opportunities such as employment and physical retail. However, there may be several near-equivalents for any given opportunity type. The simplest methods for selecting opportunities are based on nearest proximity \cite{Wachs_1973, Vickerman_1974}. Proximity methods consider that a person has a level of access to a given need as determined by that person's proximity to the closest relevant opportunity. These methods do not account for heterogeneity within an opportunity category nor for the benefits of redundancy within an opportunity category. The inverse are isocost methods wherein a person is said to have access as determined by the number of opportunities available within a given isocost polygon. This method has the drawback of not considering the differences in edge traversal cost for \gls{od} pairs within the isocost region. These methods have been used widely \cite{Easa_1993} due to their computational lightness and form the basis for modern big-data methods such as the US DOE's Mobility Energy Productivity metric \cite{Hou_2019}.

Proximity and isocost methods are easy to compute because they treat redundancy arbitrarily. In practice, equivalent and near-equivalent opportunities compete with one-another if sufficiently proximate or if the paths required to reach them overlap \cite{Stouffer_1940}. Gravity/entropy methods \cite{Noulas_2012, Jung_2008} address this shortcoming. These methods are so called as they concern the cumulative effect of opportunities for a given origin on the basis of demand over proximity (gravity) or information content (entropy). Such methods were first formalized into a quantitative framework in 1959 \cite{Hansen_1959} as a generalization of previous methodology for quantifying the efficiency of urban land use. Gravity/entropy methods define accessibility as the intensity of the possibility for interaction. Implicit in the formulation of gravity/entropy methods is that every opportunity has some effect on every individual, even if negligible, and the effect of any one opportunity is determined by its network position.

Proximity and gravity/entropy methods rely on the assumption that traversal cost is the primary factor determining individuals decision to select one opportunity from among a set of similar entities. While this is certainly true if the difference in traversal cost is large enough it is not, altogether, obvious what the threshold of disambiguation is for a given individual. Thus, researchers have proposed to use Discrete Choice Modeling \cite{Ben_Akiva_1985} to explain revealed choices wherein ease-of-access is one of several possible factors in determining the utility of a given opportunity for an individual \cite{Cevero_1995, Shen_1998, Karst_2003}.

There are, thus, a variety of methods which can be used to quantify the accessibility of a given region with varying computational and data requirements. The relationship between land-use, transportation, and demography is circular rather than linear. Which method one chooses for an analysis reflects the scope and purpose of that analysis. Definition of scope can be difficult and can lead to self-defeating policies \cite{Handy_1996}. This study is concerned with the effects of electrification on long-trip accessibility for road vehicle users. This scope simplifies opportunity selection. It is necessary that a transportation system provide for access between large population centers within a region of interest and to those in adjacent regions.  This study is focused on non-routine regional travel rather than routine local travel as this is where supply infrastructure becomes important. It should be noted that the method is valid for all travel scales. Methodology is developed in the following section.

\section*{Methods}

\subsection*{Vehicle Reduced Sub-Network Graph}

Critical to this analysis is the definition of the \gls{rsng} for a given vehicle. Powered vehicles are range-limited due to the finite capacity of their \glspl{ess}. In order to traverse the edge defined by an \gls{od} pair whose energy requirement is greater than the capacity of the \gls{ess} a vehicle must stop at a supply station. The above applies equally to road vehicles of all powertrain types, the difference being the qualities of the respective supply networks which are neither equivalent nor isomorphic.

For a network $G = \{V, E\}$ where $V$ is the set of nodes and $E$ is the set of edges, a reduced sub-network $\hat{G} = \{\hat{V}, \hat{E}\}$ can be computed where $\hat{V} \subseteq V$ and $\hat{E}$ is the set of paths between all nodes in $\hat{V}$. In other words, the cardinality of $V$ is reduced but the relationships between the \gls{od} pairs in $\hat{V}$ are maintained by considering multi-edge paths which contained nodes not in $\hat{V}$ as single edges. For a road vehicle with \gls{ess} capacity $C$ located at an origin node $v_i \in O \subseteq V$ $\hat{V}$ contains $O$, supply nodes in $S$, and opportunity nodes $D \subseteq V$. $\hat{E}$ contains all resulting edges where the edge traversal cost $f((o, d)) \leq C$. Where a destination is in range from the vehicle starting position a direct path will be seen on the \gls{rsng} and will be the shortest path. Otherwise, an indirect path utilizing at least one supply station may be the shortest path. An example source and equivalent sub-network are shown in Figure \ref{fig:reduced_sub_network_graph}.

\begin{figure}[H]
	\centering
	\includegraphics[width = \linewidth]{figs/reduced_sub_network_graph.png}
	\caption{Example original and reduced sub-network graphs. The original graph is a lattice where all edges are valued at 1. The reduced network contains all paths less than or equal to 2.5 between selected nodes as edges.}
	\label{fig:reduced_sub_network_graph}
\end{figure}

\subsection*{Models}

The method requires several models which serve to inform optimal routing. Vehicular, infrastructural, and individual models are described below.

\subsubsection*{Vehicles}

Vehicles effect routing due to their range limits and the time required to supply them. Vehicular parameters are listed in Table \ref{tab:param_veh}.

\begin{table}[H]
	\centering
	\caption{Vehicle Parameters for Routing}
	\label{tab:param_veh}
	\begin{tabular}{|C{\linewidth*3/8}|C{\linewidth*3/8}|C{\linewidth*2/8}|}
		\hline Parameter & Description & Unit \\
		\hline \gls{ess} Capacity & Accessible energy storage capacity for vehicle & [kWh] \\
		\hline Energy Consumption & Energy required to move the vehicle a given unit of distance & [kJ/km] \\
		\hline Energizing Rate & Rate at which energy can be added to the \gls{ess} at a supply location & [kW] \\
		\hline DC Charge Limits & Lower and upper \gls{soc} bounds for DC charging & [-] \\
		\hline
	\end{tabular}
\end{table}

\subsubsection*{Supply Stations}

Supply infrastructure impacts routing by providing the ability for trips greater than a vehicle's usable range to be completed. Utilizing a supply station resets the vehicle's \gls{soc} to a set level but costs money and time. The cost of the energy supplied is a station level parameter and may vary by location and time. The time added due to energizing is a function of the vehicle and supply equipment maximum energizing rate, the total energy added, and the time spent prior to energizing due to queuing if no equipment is immediately available. The amount of time that a vehicle can expect to queue is a function of the arrival rate of vehicles at a given station, the time taken to energize by those vehicles, the number of supply equipment at the station, and the functionality rate of said equipment. Expected queuing time distributions for different arrival rates and numbers of equipment for DC \gls{ev} chargers are shown in Figure \ref{fig:expected_delay}.

\begin{figure}[H]
	\centering
	\includegraphics[width = \linewidth]{figs/expected_delay.png}
	\caption{Expected queuing time at charging stations. Low arrival rate is 10 - 60 minutes between arrivals. High arrival rate is 1 - 10 minutes between arrivals. All charge rates are 80 kW and charge events are 60 $\pm$ 15 kWh.}
	\label{fig:expected_delay}
\end{figure}

Expected queuing time is highly determined by variable factors like arrival rate, which must be estimated, and numbers of usable equipment, which is a function of installed equipment and reliability. The amounts of energy purchased are less variable station-to-station for the same powertrain type. The queuing times in this study are computed using the expected waiting time formula for a M/M/c queue. This approximation is reflective of reality as long trips will begin well before there can be certainty about queues at distant stations. Supply station parameters are listed in Table \ref{tab:param_supply}.

\begin{table}[H]
	\centering
	\caption{Supply Station Parameters for Routing}
	\label{tab:param_supply}
	\begin{tabular}{|C{\linewidth*3/8}|C{\linewidth*3/8}|C{\linewidth*2/8}|}
		\hline Parameter & Description & Unit \\
		\hline Arrivals Ratio & Number of arrivals at a station per servicer-hour & [-] \\
		\hline Service Time & Time taken to supply energy to vehicles & [s] \\
		\hline Servicers & Number of chargers/pumps at a station which can be used simultaneously & [-] \\
		\hline Functionality Rate & Percentage of the time that a given servicer will be usable & [-] \\ 
		\hline
	\end{tabular}
\end{table}

All of the parameters in Table \ref{tab:param_supply} are sampled from random distributions. Arrivals Ratio is sampled from a normal distribution centered on 1 with a standard deviation of .5. Service Time is derived from the station max charging rate and charge event energy sampled from a normal distribution centered on 60 kWh with a standard deviation of 15 kWh. The number of servicers for a given station is taken from data or inputted as a parameter as is the functionality rate.

\subsubsection*{Drivers}

Given the same physical circumstances, different drivers will evaluate route costs differently. In a basic sense, drivers will weight several factors such as time, money, distance, and complexity differently. Where any important factor is not known precisely drivers will consider a range of outcomes and decide based on an expectation. Driver risk attitude concerns what range of outcomes will be used to compute expected cost. Risk attitude is modeled using a superquantile risk function defined as

\begin{equation}
	S(D, p_0, p_1) = \frac{1}{p_1 - p_0}\int_{p_0}^{p_1}Q(D, \alpha)\ d\alpha \label{eq:superquantile}
\end{equation}

\noindent where $D$ is a distribution, $p_0$ and $p_1$ are the boundaries of the range of probabilities considered in the expectation, and $Q$ is the quantile function of $D$. The superquantile is, thus, the mean value of a distributed quantity within a range of probability. $S(D, 0, 1)$ reduces to the mean of $D$. Drivers with an aggressive risk attitude will consider a low range of probabilities. Drivers with a neutral attitude will consider a central range of probabilities. Drivers with a cautious attitude will consider a high range of probabilities. Driver parameters are listed in Table \ref{tab:param_driver}.

\begin{table}[H]
	\centering
	\caption{Supply Station Parameters for Routing}
	\label{tab:param_driver}
	\begin{tabular}{|C{\linewidth*3/8}|C{\linewidth*3/8}|C{\linewidth*2/8}|}
		\hline Parameter & Description & Unit \\
		\hline Route Cost Weights $W$ & Set of multipliers for route costs to be used in computation of weighted sum & [-] \\
		\hline Probability Range $(p_0, p_1)$ & Range of probabilities for superquantile function & [-] \\
		\hline
	\end{tabular}
\end{table}

\subsection*{Stochastic Optimal Routing}

The purpose of stochastic optimal routing is to find the expected "shortest-paths" from a given origin $o \in V$ to a set of destinations $D \in V$ on a network defined by graph $G = \{V, E\}$. The output of the routing algorithm is tree $P$ containing the optimal-feasible paths from the origin to all destinations as edges. The objective of routing between $i$ and $j$ is

\begin{equation}
	\min_{U \in \overline({U}_{i,j})}\ \mathbb{E}[J(S_0, U)]
\end{equation}

where

\begin{equation}
	J(U) = \sum_{k = 0}^M w_k\phi_k(s_{0,k}, U)
\end{equation}

s.t.

\begin{equation}	
	b^w_l \leq \mathbb{E}\left[\int_0^t \Phi_w(S_0, U)dt\right] \leq b^w_u\quad \forall t \in T
\end{equation}

\noindent where $S_0$ is the vector of route states, $U$ is a control (route) between $i$ and $j$, $\overline{U}_{i,j}$ is the set of possible routes between $i$, and $j$, $\Phi_w$ is the cost function for route weight $w \in W$, and $\mathbb{E}$ denotes an expectation and is computed as in \eqref{eq:superquantile}. The optimal routes may be found using the Dijkstra or Bellman-Ford algorithm. In either case route states $S$ are initialized and stored as vectors containing $N$ discreet variables. An empirical distribution $D$ for a state vector can be computed from a histogram of the values. States can be changed at nodes and edges. Routes are considered feasible if state expectations remain within set bounds. Comparison between routes is performed using cost expectation.

\subsection*{Long-Trip Accessibility Metric}

In the presented framework accessibility can be computed in general for a region or in specific for a given origin within the region. In either case, the regional \gls{rsng}, travel mode, mode-specific infrastructure, and traveler must be modeled. For region $R$ with a set of $N$ selected nodes $O = \{O_1, O_2, \dots, O_N\}$, a corresponding set of importance weights $W = \{W_0, W_1, \dots, W_N\}$, and optimal path trees $P = \{P_1, P_2, \dots, P_N\}$, accessibility metric $A$ is the weighted mean

\begin{equation}
	A(R) = \frac{1}{N^2}\sum_{i = 0}^{N} \sum_{j = 0 }^{N} W_iW_jP_{i,j} \label{eq:a}
\end{equation}

\noindent where $P_{i,j}$ is the cost of the optimal path from $O_i$ to $O_j$. The specific accessibility metric $A(R, o)$ can be computed for region $R$ and origin $o$ by setting $W = \{W_o\}$ and $P = \{P_o\}$. Thus, a direct comparison in accessibility between regions, origins, travel modes, travelers, infrastructure deployments, or any of those mentioned in any combination can be conducted.

Long-trip accessibility metric $A$ is of the proximity type concerned with the expected cost to transit edges between pre-selected O/D pairs. In contrast with accessibility studies focused on routine travel behavior, this study is focused on non-routine long trips. It is assumed that travelers, when considering non-routine long trips have a specific destination in mind and would not consider equivalent and equidistant cities as fungible. Rather, it is assumed that, having determined to travel to a given location, the traveler will then decide on mode.

A transportation accessibility metric should have land-use, transportation, temporal, and individual components \cite{Karst_2003}. Metric $A$ contains all of these components. The land-use within a region has two principle effects on $A$. First, for a multi-city region, specific accessibility should vary within the region with peripheral cities experiencing worse regional access than central cities. Second, geographically large and/or sparse regions should experience worse accessibility than compact regions. The transportation infrastructure determines the efficacy of various modes. Where only vehicular travel is concerned, the mode choice is reduced to vehicle choice. Vehicle mode-specific infrastructure is not necessarily the same for all vehicles of a given fuel type but is necessarily different between fuel types thus separating vehicular modes. The temporal component of $A$ arises from the schedules of public transportation services and road traffic patterns. Finally the individual component is the traveler risk attitude as modeled using superquantile function parameters and route cost weights.

\section*{California Case Study}

\subsection*{Background}

The state of California is geographically large and contains major population centers distributed across the state. Major road transportation corridors form connections between the state's population centers and with population centers in adjacent states. This case study concerns the long-trip accessibility of California's road transportation network. For the purposes of analysis, 15 important cities in California and adjacent states were selected. Non-California locations are represented by the most-likely departure locations at the California state line. These locations are enumerated in Table \ref{tab:locations}.

\begin{table}[H]
	\centering
	\caption{Locations Considered for Long Trip Accessibility}
	\label{tab:locations}
	\begin{tabular}{|C{\linewidth * 1 / 3}|C{\linewidth * 2 / 3}|}
		\hline Index & Location \\
		\hline 0 & Crescent City \\
		\hline 1 & Yreka \\
		\hline 2 & Redding \\
		\hline 3 & Chico \\
		\hline 4 & Reno (State Line) \\
		\hline 5 & Sacramento \\
		\hline 6 & Stockton \\
		\hline 7 & San Francisco \\
		\hline 8 & San Jose \\
		\hline 9 & Fresno \\
		\hline 10 & Las Vegas (State Line) \\
		\hline 11 & Bakersfield \\
		\hline 12 & Los Angeles \\
		\hline 13 & Phoenix (State Line) \\
		\hline 14 & San Diego \\
		\hline
	\end{tabular}
\end{table}

Locations were chosen either because they are, themselves, important destinations in California or because they are the closest points in California to important destinations in neighboring states. For long trips, \glspl{bev} will rely on DC charging stations. The locations of all DC charging stations in California were scraped from AFDC \cite{afdc_2023}. California's DC charging stations include proprietary stations such as Tesla Superchargers and the Rivian Adventure network as well as non-proprietary stations such as those operated by ChargePoint, Electrify America, eVgo, and others. The selected locations and DC charging stations are mapped in Figure \ref{fig:california_atlas}.

\begin{figure}[H]
	\centering
	\includegraphics[width = \linewidth]{figs/California_Places_Chargers.png}
	\caption{Selected locations and DC charging stations for California case study}
	\label{fig:california_atlas}
\end{figure}

The proprietary and non-proprietary networks in California are neither equivalent nor isomorphic. There are 403 Tesla and 78 Rivian DC charging stations in the state as compared to 1,425 non-proprietary DC charging stations. The non-Tesla chargers overwhelmingly use combination CCS/ChaDeMo plugs as defined by SAE J1772 \cite{SAE_J1772} which reflect the ports on the overwhelming number of non-Tesla \glspl{bev}. By contrast, Tesla chargers and vehicles use the NACS standard as defined by SAE J3400 \cite{SAE_J3400}. The Tesla and non-Tesla systems are increasingly interoperable with the aid of adapters but should be considered separately in the present. Tesla drivers use Tesla DC chargers almost exclusively \cite{Visaria_2022} and CCS/ChaDeMo to NACS adapters are more common than their counterparts. The Rivian Adventure network is technically interoperable with other J1772 vehicles but is set aside for the exclusive use of Rivian vehicles. The purpose of the Rivian Adventure network is to allow for Rivian vehicles to charge in remote locations and is not intended to be the only network used by them.

The difference between the Tesla DC charging network and the non-proprietary network extends from function to form. Built out as an investment to entice sales of Tesla vehicles and, until recently, exclusive to them, the Tesla network is technically superior with higher rate and more reliable chargers \cite{Rempel_2023, Kozumplik_2022}. Non-proprietary networks have, so far, been subsidy driven \cite{Gamage_2023} and have responded to incentives to form a widely distributed but thin presence. A stark contrast is seen when examining the ratio of chargers to stations. In California there are 403 Tesla DC charging stations with a total of 6,277 DC chargers for an average of 15.6 chargers per station. Among non-proprietary networks there are a total of 1,425 stations with 3,667 chargers for an average of 2.6 per station. The distributions of chargers per station for Tesla and non-proprietary networks are shown in Figure \ref{fig:network_histograms}. As demonstrated in Figure \ref{fig:expected_delay}, the number of chargers at a station has a major effect on expected delay time as well as determining the importance of individual equipment failure. 

\begin{figure}[H]
	\centering
	\includegraphics[width = \linewidth]{figs/California_Charger_Network_Survival_Functions.png}
	\caption{Survival functions for charger networks in California. Top panel shows survival function for chargers per station. Bottom panel shows survival function for mean distance to three nearest stations from a given station.}
	\label{fig:network_histograms}
\end{figure}

California \glspl{icev} utilize a third and completely separate network of supply stations. There are estimated to be over 8,000 gasoline stations in California \cite{CEC_2022} and these are widely and proportionally distributed. Because no public database for the locations of gasoline stations in the state exists, and due to their ubiquity it is assumed in this study that \gls{icev} driver optimal paths will not be effected by fueling station availability and that such stations will be available wherever needed. For this reason, \glspl{icev} are assumed to take the "direct" path between cities where \glspl{bev} need to find optimal paths on their \glspl{rsng}.

\subsection*{Example}

Consider, as an example, the following four scenarios for a driver based out of Fresno:


\begin{compactenum}
	\item Risk neutral driver using a generic \gls{icev} with an \gls{ess} capacity of 550 kWh and an efficiency of 2700 kJ/km.
	\item Risk neutral driver using a Tesla Model 3 with an 80 kWh battery and efficiency of 536.4 kJ/km capable of charging at a max rate of 170 kW. The driver uses the Tesla DC station network exclusively and this network has high charger up-time (97\%). The driver only charges to 80\% \gls{soc} for DC events.
	\item Risk-cautious driver using a Chevrolet Bolt EV with an 65 kWh battery and efficiency of 626.5 kJ/km capable of charging at a max rate of 55 kW. The driver uses various non-proprietary networks and these networks have low charger up-time (75\%). The driver only charges to 80\% \gls{soc} for DC events.
	\item Risk-aggressive driver using a Chevrolet Bolt EV with an 65 kWh battery and efficiency of 626.5 kJ/km capable of charging at a max rate of 55 kW. The driver uses various non-proprietary networks and these networks have low charger up-time (75\%). The driver only charges to 80\% \gls{soc} for DC events.
\end{compactenum}

Each of these drivers will have a different experience of California's road transportation system for long trips. The neutral-expectations of time to reach each of the selected locations in the case study are provided in Table \ref{tab:scenarios} (driver is required to arrive at each destination with  at least 20\% \gls{soc}).

\begin{table}[H]
	\centering
	\caption{Neutral expectation of hours to locations from Fresno for example scenarios.}
	\label{tab:scenarios}
	\begin{tabular}{|C{.17\linewidth}|C{\linewidth * 1 / 5}|C{\linewidth * 1 / 5}|C{\linewidth * 1 / 5}|C{.23\linewidth}|}
		\hline Index & \gls{icev} & Model 3 Neutral & Bolt Cautious & Bolt Aggressive \\
		\hline 0 & 8.79 & 9.28 & 15.58 & 13.66 \\
		\hline 1 & 6.73 & 7.07 & 13.31 & 10.04 \\
		\hline 2 & 5.28 & 5.60 & 8.78 & 8.35 \\
		\hline 3 & 4.40 & 4.54 & 7.54 & 5.96 \\
		\hline 4 & 4.63 & 4.74 & 7.85 & 6.34 \\
		\hline 5 & 2.79 & 2.82 & 4.01 & 4.17 \\
		\hline 6 & 2.09 & 2.09 & 2.09 & 2.09 \\
		\hline 7 & 3.08 & 3.18 & 4.34 & 4.50 \\
		\hline 8 & 2.71 & 2.71 & 2.71 & 2.71 \\
		\hline 9 & 0.00 & 0.00 & 0.00 & 0.00 \\
		\hline 10 & 5.86 & 6.13 & 8.88 & 9.05 \\
		\hline 11 & 1.64 & 1.64 & 1.64 & 1.64 \\
		\hline 12 & 3.32 & 3.32 & 4.72 & 4.73 \\
		\hline 13 & 6.80 & 7.31 & 11.21 & 10.50 \\
		\hline 14 & 5.29 & 5.47 & 8.41 & 8.32 \\
		\hline
	\end{tabular}
\end{table}

The example drivers have very different experiences and perceived experiences. While the \gls{icev} driver can take the "direct" path (not deviating to find a station), the \gls{bev} drivers have to deviate from the "direct" path and require substantial time to charge. However, there is quite significant disparity within the set of proposed \gls{bev} drivers. For destinations which are within full-charge range, there is no difference between any of the drivers experiences. However, due to the lower range, lower max charge rate, and lower infrastructure reliability for the Bolt, the differentials between it and the Model 3 can be tremendous, especially for proximate and remote locations. It is worth noting, also, that the \gls{icev} and Model 3 are often able to take more direct routes. Table \ref{tab:scenarios_nc} shows the optimal route times with charging/fueling times removed.

\begin{table}[H]
	\centering
	\caption{Neutral expectation of hours to locations from Fresno for example scenarios without charging/fueling time.}
	\label{tab:scenarios_nc}
	\begin{tabular}{|C{.17\linewidth}|C{\linewidth * 1 / 5}|C{\linewidth * 1 / 5}|C{\linewidth * 1 / 5}|C{.23\linewidth}|}
		\hline Index & \gls{icev} & Model 3 Neutral & Bolt Cautious & Bolt Aggressive \\
		\hline 0 & 8.76 & 8.80 & 9.58 & 8.76 \\
		\hline 1 & 6.71 & 6.75 & 7.31 & 6.72 \\
		\hline 2 & 5.26 & 5.34 & 5.78 & 5.33 \\
		\hline 3 & 4.39 & 4.42 & 4.54 & 4.39 \\
		\hline 4 & 4.63 & 4.63 & 4.85 & 4.67 \\
		\hline 5 & 2.78 & 2.82 & 2.81 & 2.81 \\
		\hline 6 & 2.09 & 2.09 & 2.09 & 2.09 \\
		\hline 7 & 3.07 & 3.07 & 3.07 & 3.07 \\
		\hline 8 & 2.71 & 2.71 & 2.71 & 2.71 \\
		\hline 9 & 0.00 & 0.00 & 0.00 & 0.00 \\
		\hline 10 & 5.86 & 5.89 & 5.88 & 5.86 \\
		\hline 11 & 1.64 & 1.64 & 1.64 & 1.64 \\
		\hline 12 & 3.32 & 3.32 & 3.32 & 3.32 \\
		\hline 13 & 6.79 & 6.93 & 6.81 & 7.14 \\
		\hline 14 & 5.28 & 5.28 & 5.41 & 5.28 \\
		\hline
	\end{tabular}
\end{table}

The differences in total expected travel time between the \gls{icev} and Model 3 are relatively small and in the range of 5-10\% of travel time, a testament to Tesla's DC station network. Some would argue that this difference is unimportant as Tesla drivers can charge their vehicles while stopping for meals or during other natural breaks. This logic has been used to show that \glspl{bev} may approach convenience parity with \glspl{icev} in good circumstances \cite{Dixon_2020}. However, where such breaks are optional for \gls{icev} drivers, they are mandatory for \gls{bev} drivers and must be taken at specific points throughout the trip to coincide with charging. The loss of optionality must be accounted an inconvenience even if breaks would be taken in any event.

\subsection*{Experiment}

In order to gain an understanding of the effects of vehicular, infrastructural, and individual characteristics on transportation accessibility for long trips in California a designed experiment was conducted on the parameters in Table \ref{tab:exp_parameters}.

\begin{table}[H]
	\centering
	\caption{Parameters and levels for designed experiment}
	\label{tab:exp_parameters}
	\begin{tabular}{|C{.5\linewidth}|C{.5\linewidth}|}
		\hline Parameter & Levels \\
		\hline Charger Network & [Tesla, Non-Proprietary] \\
		\hline Range/Max Charge Multiplier & [1, 1.25, 1.5] \\
		\hline Charger Reliability & [.75, .85, .95] \\
		\hline Risk Attitude & [Cautious, Neutral, Aggressive] \\
		\hline
	\end{tabular}
\end{table}

The parameters and levels selected reflect possible ways to mitigate the disparity between \glspl{bev} and \glspl{icev} and within both sets. The rationale behind the selection of levels is as follows. As mentioned previously, the Tesla and non-proprietary DC charging networks are, presently, effectively separate and unequal entities. The base vehicle models in this study are the Chevrolet Bolt and Tesla Model 3 both of which are low-end vehicles for their category with more expensive models coming with higher battery capacities and charge rates \cite{AFDC_EVs_2023}. The range of reliability rates was taken from references \cite{Rempel_2023} with an aspirational high reliability rate added. Finally, risk attitudes were modeled on the theoretical basis outlined in the Methods section using \eqref{eq:superquantile} with Cautious ($p_0 = .5,\ p_1 = 1$), Neutral ($p_0 = 0,\ p_1 = 1$), and Aggressive ($p_0 = 0,\ p_1 = .5$) drivers.

The metric used for comparison is long-trip accessibility as defined in \eqref{eq:a} where the cost used is total route time. In each case the optimal-path tree $P$ is computed using a stochastic, constrained implementation of Dijkstra's method as described in the Methods section with sample vectors of length 100. On average the difference between the accessibility score $A$ for the Model 3 and Bolt was 2.72 hours in favor of the Model 3 for the same combinations of experimental parameters. This difference is due to the superior range, charging rate, and DC charging infrastructure enjoyed by the Model 3. Linear regression was performed on the experimental parameters, interactions and results. Significant parameters ($\alpha = 0.05$) for the Bolt and Model 3 are listed in Tables \ref{tab:sp_bolt} and \ref{tab:sp_model_3} respectively.

\begin{table}[H]
	\centering
	\caption{Significant ($\alpha = 0.05$) terms from linear regression for Bolt.}
	\label{tab:sp_bolt}
	\begin{tabular}{|C{.6\linewidth}|C{.2\linewidth}|C{.2\linewidth}|}
		\hline Parameter & $\beta$ & p-value \\
		\hline {\small Intercept } & 8.557 & 0.000 \\
		\hline {\small Multiplier } & -2.127 & 0.000 \\
		\hline {\small Attitude[T.Neutral] } & 0.775 & 0.004 \\
		\hline {\small Attitude[T.Cautious] } & 1.188 & 0.000 \\
		\hline
	\end{tabular}
\end{table}

\begin{table}[H]
	\centering
	\caption{Significant ($\alpha = 0.05$) terms from linear regression for Model 3.}
	\label{tab:sp_model_3}
	\begin{tabular}{|C{.6\linewidth}|C{.2\linewidth}|C{.2\linewidth}|}
		\hline Parameter & $\beta$ & p-value \\
		\hline {\small Intercept } & 5.840 & 0.000 \\
		\hline {\small Multiplier } & -0.118 & 0.000 \\
		\hline
	\end{tabular}
\end{table}

The regression results support the intuitive conclusion that vehicle range is important in determining long-trip accessibility. A longer range vehicle will have more long-trip accessibility for a region of a given size. It is somewhat surprising that charger reliability was not significant for either vehicle/supply network combination. The lack of significance of reliability in the experimental range is due to the redundancy present in both supply networks. Where the Tesla DC charger network consists of stations with many chargers the non-proprietary network consists of smaller stations in closer proximity to one-another. Because there are more chargers total in the Tesla network and because these are redundant within stations rather than between stations the effect of non-functional chargers on user experience is lesser. A consequence of the more distributed non-proprietary network is that charger non-availability can catch a driver out and require a costly event relocation or even a tow. For this reason, driver risk attitude is more important for non-Tesla drivers for whom the trade-off between more direct routes with higher risk and less direct routes with lower risk is substantial. Overall the results indicate that the reliability of individual chargers is less important than their distribution and that the non-proprietary DC charging network is currently inferior to the Tesla DC charging network in California.

In the past, and still largely in the present, the Tesla DC charging network has been sequestered for the use of Tesla drivers. Recently, with the introduction of the NACS standard, this restriction has started to loosen. In the future, the Tesla and non-Tesla networks will function, increasingly, as one. A comparison of accessibility for each vehicle with each network independently, and the combined network is shown in Table \ref{tab:vehicles_networks}.

\begin{table}[H]
	\centering
	\caption{Accessibility for Bolt and Model 3 with neutral driver for various networks.}
	\label{tab:vehicles_networks}
	\begin{tabular}{|C{.4\linewidth}|C{.3\linewidth}|C{.3\linewidth}|}
		\hline Network & Bolt & Model 3 \\
		\hline Non-Proprietary & 9.49 & 6.32 \\
		\hline Tesla & 7.63 & 5.85 \\
		\hline Combined & 7.63 & 5.85 \\
		\hline
	\end{tabular}
\end{table}

The results show that the experience of Bolt drivers is improved by being able to use the Tesla DC charging network but does not redress the gap to Teslas using the Tesla network. Tesla drivers experiences are worsened by being only able to use the non-proprietary network but not to the same extent as those of Bolt drivers. It is notable that for the combined network both vehicles made exclusive use of Tesla DC charging stations. Given the demonstrated superiority of the Tesla network one might suppose that its opening will markedly improve the experience of non-Tesla drivers while failing to improve that of Tesla drivers. In fact, it may deteriorate Tesla driver experience due to higher demand at Tesla DC charging stations.

That the Tesla network is preferable to the non-proprietary network is due to redundancy. The Tesla network in California has roughly twice the number of chargers as the non-proprietary network with less than 30\% of the stations. Redundancy is not trivial to quantify for a network. Tesla chargers reinforce Tesla chargers within station but the distances between stations are large. Non-proprietary chargers reinforce between stations with smaller distances between. There is reason to suspect that even if the number of non-proprietary chargers were doubled but the distribution remained similar that Tesla's network would be preferable. The difference is informational. Where a driver finds a queue at a small station, that driver has uncertain information about the availability of chargers at proximate stations. Further, should that driver relocate to another station and find a similar or larger queue there is no mechanism to return to the vacated queue position at the previous station. This issue is more relevant for long-trips tahn it is for routine charging as the former generates less flexible charging demand. Policy makers should carefully consider which type of redundancy their policies encourage and what effects this will have. Quantifying the impacts of this informational issue is out of scope for this study and will be the subject of future research.

\section*{Conclusions}

Transportation accessibility is a very useful framework for policymakers and planners as it allows for the simultaneous consideration of land-use and transportation. For long trips, the land use component becomes exogenous in the short and medium term and the differentiating factor is transportation. Consideration of long road trips plays a role in vehicle selection out of proportion with their regularity encouraging continued \gls{icev} retention at the individual and household level. Quantifying the disadvantages of \glspl{bev} compared to \glspl{icev} is critical in addressing this issue. The framework and methodology developed are a valuable tool for th origination and evaluation of future policy.

In this study a framework for accessing long-trip accessibility for road vehicles in a region is developed. Using this framework and current data, a case study for the state of California was conducted comparing long-trip accessibility between \glspl{icev} and \glspl{bev} and between different \glspl{bev}. Results show that \glspl{icev} retain an advantage over \glspl{bev} of all types but this advantage is marginal for Tesla vehicles but major for low-end non-Tesla vehicles. the major difference is access to the Tesla DC charging network which is larger and differently structured compared to the non-proprietary networks. Specifically, the Tesla DC fast charging network is concentrated in large stations providing redundancy within station where the non-proprietary networks consist of distributed small stations providing redundancy between stations. The different approaches between Tesla and non-proprietary networks are due to the economic circumstances under which they developed. Policymakers and planners should note the differences between them and the effects these have on long-trip accessibility in considering new incentive programs.


 

\newpage

\printbibliography

\end{multicols}

\end{document}

%\subsection*{Random Graph Example}
%
%The optimal routing method in this study accounts for factors which influence road vehicle accessibility and derive from vehicular, infrastructural, and behavioral parameters. Important vehicular parameter for accessibility is range which determines which locations can be reached with and without charging. Important infrastructural parameters such as the number and locations of charging stations, usability rates, and expected delay times. Important behavioral parameters include risk attitude parameters as in \eqref{eq:superquantile}. These factors influence routing individually and via interactions as demonstrated on a randomly generated graph.
%
%A random graph of 100 nodes, 15 containing charging stations within a 100 km by 100 km square was generated with a random node selected as the origin. Edge probabilities were defined relative to a characteristic distance using
%
%\begin{equation}
%	P(e) = \exp(-L(e)/d)
%\end{equation}
%
%where $e$ is an edge, $L(e)$ is the distance of edge $e$, and $d$ is a characteristic distance set to 200 km for this example. In this example the vehicle has a range of 460 km and each charging station has one charger and a low vehicle arrival rate. Figure \ref{fig:interactions} show how $R$ is effected by different usability rates and risk attitudes.
%
%\end{multicols}
%
%\begin{figure}[H]
%	\centering
%	\begin{subfigure}{.4\linewidth}
	%		\centering\includegraphics[width = \linewidth]{figs/interactions_ha.png}
	%		\captionsetup{width=.8\linewidth}
	%		\caption{High charger usability and aggressive risk attitude}
	%	\end{subfigure}%
%	\begin{subfigure}{.4\linewidth}
	%		\centering\includegraphics[width = \linewidth]{figs/interactions_hc.png}
	%		\captionsetup{width=.8\linewidth}
	%		\caption{High charger usability and cautious risk attitude}
	%	\end{subfigure}
%	\begin{subfigure}{.4\linewidth}
	%		\centering\includegraphics[width = \linewidth]{figs/interactions_la.png}
	%		\captionsetup{width=.8\linewidth}
	%		\caption{Low charger usability and aggressive risk attitude}
	%	\end{subfigure}%
%	\begin{subfigure}{.4\linewidth}
	%		\centering\includegraphics[width = \linewidth]{figs/interactions_lc.png}
	%		\captionsetup{width=.8\linewidth}
	%		\caption{Low charger usability and cautious risk attitude}
	%	\end{subfigure}
%	\caption{Single-origin expected travel time trees with varying charger reliability and driver risk tolerance}
%	\label{fig:interactions}
%\end{figure}
%
%\begin{multicols}{2}
%
%As the example shows, it is possible for the impact of several factors to be additive when made disadvantageous simultaneously. Where the driver has an aggressive risk attitude, chargers are reliable, or both, accessibility is high. Where neither is the case accessibility is low. The differential can be dramatic and this highlights the importance of considering vehicular, infrastructural, and behavioral factors simultaneously to get a complete picture.